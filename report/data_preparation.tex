\chapter{Preparação Dos Dados}

\section{Levantamento das Possíveis Features Relevantes}
De posse das informações apresentadas nos capítulos anteriores é possível definir o problema que esse trabalho se propõe a resolver e, em linhas gerais, como ele será abordado.

O objetivo desse trabalho é a partir dos registros de um servidor DNS, acusar quais máquinas na rede são suspeitos de pertencer a uma botnet e merecem atenção para uma investigação. A caracterização dos IPs será feita através da análise das características da iteração que seriam

\begin{itemize}
\item Quantidade de consultas a domínios com sufixos suspeitos;
\item Quantidade de consultas a domínios com alta quantidade de números;
%\item Quantidade de domínios consultadas com string legível pequena;
\item Quantidade de consultas a domínios com baixo grau no Alexa;
\item Quantidade total de consultas realizadas;
\item Média do comprimento de domínios consultados;
\item Desvio Padrão dos comprimentos dos domínios consultados;
\item Média do intervalo entre as consultas;
\item Desvio padrão dos intervalos entre consultas;
\item Total de consultas que resultaram em NXDOMAIN;
\item Porcentagem de consultas que resultaram em NXDOMAIN;
\item Quantidade de consultas para cada tipo de DNS;
\item Porcentagem de consultas para cada tipo de DNS;
\item Tamanho do menor clique que a máquina participa;
\item Tamanho médio dos cliques em que a máquina participa.
\end{itemize}

Após o levantamento de todas as máquinas da rede, é feita a construção do modelo probabilístico gaussiano e de acordo após a escolha da probabilidade mínima dos exemplos normais, o sistema estará pronto para decidir se uma nova máquina poderia pertencer ou não a uma botnet.
