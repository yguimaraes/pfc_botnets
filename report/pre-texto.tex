%%
%
% ARQUIVO: pre-texto.tex
%
% VERSÃO: 1.0
% DATA: Maio de 2016
% AUTOR: Coordenação de Trabalhos Especiais SE/8
% 
%  Arquivo tex para a criação da parte pré-textual do documento de Projeto de Fim de Curso.
%
%%


% -----
% PÁGINA DE CAPA DO DOCUMENTO DE PFC
% -----
\makecapa

% -----
% PÁGINA DE TÍTULO DO PFC
% -----
\prepareadvisors
\maketitle

% -----
% PÁGINA DE CRÉDITOS DO DOCUMENTO DE PFC
% -----
\makecredits

% -----
% PÁGINA DE FOLHA DE ASSINATURAS
% -----
\preparemembers
\approvalpage

% -----
% PÁGINA DE DEDICATÓRIA (OPCIONAL, ie. pode remover toda a página)
% -----
%%% DEDICATÓRIA - PREENCHER...
%\dedicatoria{%
%Ao Instituto Militar de Engenharia, alicerce da minha formação e aperfeiçoamento.
%}%
%\makededication

% -----
% PÁGINA DE AGRADECIMENTOS (OPCIONAL, ie. pode remover toda a página)
% -----
%%% AGRADECIMENTOS - PREENCHER...
\agradecimentos{%
Ao Professor Orientador Sergio dos Santos Cardoso Silva por sempre exigir o nosso melhor, pelo seu apoio e conhecimento compartilhado.\\
\indent
Aos nossos amigos e colegas que sempre estiveram do nosso lado, nos auxiliando a superar as dificuldades e compartilhando nossos momentos de felicidade.\\
\indent
As nossas famílias que sempre nos deram força e entenderam as ausências necessárias para que percorrêssemos essa longa, porém gratificante jornada.  \\
}%
\makethanks

% -----
% PÁGINA DE EPÍGRAFE (OPCIONAL, ie. pode remover toda a página)
% -----
%%% EPÍGRAFE - PREENCHER...
\epigrafe{%
Se é previsto que uma máquina seja infalível, ela não pode ser também inteligente.
}%
\autorepigrafe{%    %% Se não tem autor, coloque "Anônimo"
Alan Turing
}%
\makeepigraph

% -----
% PÁGINA DE SUMÁRIO
% -----
\tableofcontents

% -----
% PÁGINAS DE LISTAS DE FIGURAS E DE TABELAS
% se o documento de PFC não possui figuras e/ou tabelas, REMOVA O COMANDO CORRESPONDENTE
% -----
\listoffigures
\listoftables

% -----
% PÁGINA DE LISTA DE SIGLAS
% se o documento de PFC não possui siglas, REMOVA TODA A PÁGINA
% -----
%%% SIGLAS - PREENCHER...
\acronimo{C\&C}{Comando e Controle}
\acronimo{IRC}{Internet Relay Chat}
\acronimo{HTTP}{HyperText Transfer Protocol}
\acronimo{P2P}{Peer-to-peer}
\acronimo{IDS}{Intrusion Detection System}
\acronimo{DNS}{Domain Name System}
\acronimo{CDCiber}{Centro de Cibernética}
\acronimo{IDE}{Integrated Development Environment}
\acronimo{FBE}{Fluxo Básico de Eventos}

\listofnicks

% -----
% PÁGINA DE LISTA DE ABREVIATURAS
% se o documento de PFC não possui abreviaturas ou símbolos, REMOVA TODA A PÁGINA
% -----
%%% ABREVIATURAS - PREENCHER...
%\abreviatura{Ja}{jacobiano}
%\abreviatura{JS}{fluxo secundário (difusivo)}
%\abreviatura{M}{número de Mach}

%%% SÍMBOLOS - PREENCHER...
%\simbolo{$\Phi$}{termo de dissipação viscosa}
%\simbolo{$\Gamma$}{coeficiente de difusão efetivo}
%\simbolo{$\alpha$}{fator de sub-relaxação}
%\simbolo{$\phi$}{variável dependente da equação diferencial geral}

%\listofsymbols

% -----
% PÁGINA DE RESUMO
% -----
%%% RESUMO - PREENCHER...
\resumo{%
As botnets são uma ameaça cibernética que já causaram muito prejuízo financeiro. Segundo \citet{silva2013botnets}, em 2006 somou-se cerca de 13.2 bilhões de dólares de prejuízos diretos. Essa ameaça utiliza computadores infectados para realizar atividades fraudulentas como servir páginas piratas para roubar informações sensíveis, enviar \textit{spam} para usuários comuns e enviar sucessivas requisições para derrubar servidores. Por ser uma atividade ilegal, os criminosos realizam a comunicação entre as máquinas de modo oculto que força essa comunicação a ter comportamentos que divergem do padrão de uma comunicação legítima.\\
\indent
Baseado nessa premissa, esse projeto se propõe em desenvolver um sistema de detecção de máquinas com padrão de comunicação suspeito com um dado servidor DNS. Essa detecção pode ser feita através da análise do resultado de algoritmos de Aprendizado de Máquina da classe de Agrupamento, ou até mesmo analisando gráficos que tem como eixos as características do padrão de comunicação das máquinas. No uso dos algoritmos de agrupamento, espara-se reduzir a quantidade máquinas para que a análise seja feita apenas nas que apresentarem comportamente suspeito. \\
\indent
Dentro do contexto Exército Brasileiro, o projeto auxiliará a Inteligência do Exército Brasileiro na prevenção de ataques por botnets, uma vez que esse sistema aspira fazer parte do raciocinador para um sistema que desarticula botnets. Mais especificamente, será uma ferramenta que filtrará o fluxo da rede para possibilitar que o usuário consiga dar atenção às máquinas de fato suspeitas.
}%
\makeresumo

% -----
% PÁGINA DE ABSTRACT
% -----
%%% ABSTRACT - PREENCHER...
\abstract{%
The botnets are a cyber threat that already brought plenty of money dispend. According to \citet{silva2013botnets}, in 2006 it summed around US \$13.2 billion of direct financial loss. This threat uses infected computers to perform fraudulent activities, such as serving pirated sites to steal sensible information, sending spam to common users and sending successive requests to get servers down. Since it is an illegal activity, the criminals manage the communication between the machines in a hidden way that forces that communication to have behaviours that diverges from the legitimate communication pattern. \\
\indent
Based on that premise, this project proposes to develop a system to detect machines with suspect communication patterns with a given DNS server. This detection can be done by analysing Machine Learning algorithms from Clustering class results, or even analyzing graphs which have as axis the machines communication patterns features. Within the identified clusters, it is expected that some will be small, in which the machines suspected to belong to botnets and that deserve to be investigated are expected to be.\\
\indent
Inside the Brazilian Army context, the project will help the Brazilian Army Intelligence preventing botnet attacks, once this system aspires to be a piece of an integrated botnet desarticulating system. More specifically, it will be a tool that filters the network stream in order to make possible to an user pay attention to the actual suspect machines.
}%
\makeabstract
