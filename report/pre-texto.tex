%%
%
% ARQUIVO: pre-texto.tex
%
% VERSÃO: 1.0
% DATA: Maio de 2016
% AUTOR: Coordenação de Trabalhos Especiais SE/8
% 
%  Arquivo tex para a criação da parte pré-textual do documento de Projeto de Fim de Curso.
%
%%


% -----
% PÁGINA DE CAPA DO DOCUMENTO DE PFC
% -----
\makecapa

% -----
% PÁGINA DE TÍTULO DO PFC
% -----
\prepareadvisors
\maketitle

% -----
% PÁGINA DE CRÉDITOS DO DOCUMENTO DE PFC
% -----
\makecredits

% -----
% PÁGINA DE FOLHA DE ASSINATURAS
% -----
\preparemembers
\approvalpage

% -----
% PÁGINA DE DEDICATÓRIA (OPCIONAL, ie. pode remover toda a página)
% -----
%%% DEDICATÓRIA - PREENCHER...
%\dedicatoria{%
%Ao Instituto Militar de Engenharia, alicerce da minha formação e aperfeiçoamento.
%}%
%\makededication

% -----
% PÁGINA DE AGRADECIMENTOS (OPCIONAL, ie. pode remover toda a página)
% -----
%%% AGRADECIMENTOS - PREENCHER...
%\agradecimentos{%
%Agradeço a todas as pessoas que me incentivaram, apoiaram e possibilitaram esta oportunidade de ampliar meus horizontes. \\
%\indent
%Meus familiares, cônjuge e mestres.\\
%\indent
%Em especial ao meu Professor Orientador Dr. Antonio José Reis e ao Professor Co-orientador Dr. Joel Duarte Silva, por suas disponibilidades e atenções.
%}%
%\makethanks

% -----
% PÁGINA DE EPÍGRAFE (OPCIONAL, ie. pode remover toda a página)
% -----
%%% EPÍGRAFE - PREENCHER...
%\epigrafe{%
%Sem publicação, a ciência é morta.
%}%
%\autorepigrafe{%    %% Se não tem autor, coloque "Anônimo"
%Gerard Piel
%}%
%\makeepigraph

% -----
% PÁGINA DE SUMÁRIO
% -----
\tableofcontents

% -----
% PÁGINAS DE LISTAS DE FIGURAS E DE TABELAS
% se o documento de PFC não possui figuras e/ou tabelas, REMOVA O COMANDO CORRESPONDENTE
% -----
\listoffigures
%\listoftables

% -----
% PÁGINA DE LISTA DE SIGLAS
% se o documento de PFC não possui siglas, REMOVA TODA A PÁGINA
% -----
%%% SIGLAS - PREENCHER...
\acronimo{C\&C}{Comando e Controle}
\acronimo{IRC}{Internet Relay Chat}
\acronimo{HTTP}{HyperText Transfer Protocol}
\acronimo{P2P}{Peer-to-peer}
\acronimo{IDS}{Intrusion Detection System}
\acronimo{DNS}{Domain Name System}
\acronimo{CDCiber}{Centro de Cibernética}

\listofnicks

% -----
% PÁGINA DE LISTA DE ABREVIATURAS
% se o documento de PFC não possui abreviaturas ou símbolos, REMOVA TODA A PÁGINA
% -----
%%% ABREVIATURAS - PREENCHER...
%\abreviatura{Ja}{jacobiano}
%\abreviatura{JS}{fluxo secundário (difusivo)}
%\abreviatura{M}{número de Mach}

%%% SÍMBOLOS - PREENCHER...
%\simbolo{$\Phi$}{termo de dissipação viscosa}
%\simbolo{$\Gamma$}{coeficiente de difusão efetivo}
%\simbolo{$\alpha$}{fator de sub-relaxação}
%\simbolo{$\phi$}{variável dependente da equação diferencial geral}

%\listofsymbols

% -----
% PÁGINA DE RESUMO
% -----
%%% RESUMO - PREENCHER...
\resumo{%
Botnets são uma ameaça cibernética que já trouxe muito prejuízo \citep{silva2013botnets}. Essa ameaça utiliza computadores infectados para realizar atividades fraudulentas como servir páginas piratas para roubar informações sensíveis, enviar \textit{spam} para usuários comuns e enviar sucessivas requisições para derrubar servidores. Por ser uma atividade ilegal, os criminosos realizam a comunicação entre as máquinas com comportamentos divergentes. Essa é uma questão de grande importância para o Exército Brasileiro, uma vez que em 2008 foi aprovado um decreto que determina a Defesa Cibernética um dos setores prioritários para o exército brasileiro\\
\indent
Baseado nessa premissa, esse projeto se propõe a auxiliar na detecção de máquinas que pertencem à botnets utilizando algoritmos de Aprendizado de Máquina da classe de Agrupamento. Dentro dos grupos identificados, espera-se que apareçam grupos pequenos, nos quais estarão as máquinas suspeitas de pertencerem à botnets e que devem ser investigadas. \\
\indent
Dentro do contexto Exército Brasileiro, o projeto auxiliará a Inteligência do Exército Brasileiro na prevenção de ataques por botnets. Dado que, esse sistema aspira fazer parte do raciocinador para um sistema que desarticula botnets. Mais especificamente, será uma ferramenta que filtrará o fluxo da rede para possibilitar que o usuário consiga dar atenção às máquinas de fato suspeitas. Além disso, o sistema deverá auxiliar o analista a entender o que caracterizou tais máquinas como suspeitas.
}%
\makeresumo

% -----
% PÁGINA DE ABSTRACT
% -----
%%% ABSTRACT - PREENCHER...
\abstract{%
Botnets are a cyber threat that already brought plenty of money dispend \citep{silva2013botnets}. This threat uses infected computers to perform fraudulent activities, such as serving pirated sites to steal sensible information, sending spam to common users and sending successive requests to get servers down. Because it is an illegal activity, the criminals do the communication between the machines with divergent behaviours. This matter is a great issue for the brazilian army, because in 2008 a decret was approved that makes Cyber Defence one of the army's top priorities.\\
\indent
Based on that premise, this project proposes to detect machines that are part of a botnet using Clustering, a type of Machine Learning algorithms. Within the identified clusters, it is expected that some will be small, in which the machines suspected to belong to botnets and that deserve to be investigated are expected to be.\\
\indent
Inside the Brazilian Army context, the project will help the Brazilian Army Intelligence preventing botnet attacks. This system aspires to be an important piece of an integrated botnet desarticulating system. More specifically, it will be a tool that filters the network stream in order to make possible to an user pay attention to the actual suspect machines. Besides that, the system must help the user to understand what made those machines suspect.
}%
\makeabstract
