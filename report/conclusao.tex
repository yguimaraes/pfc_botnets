\chapter{Conclusão}

Os clientes IP foram modelados segundo algoritmos de agrupamento apresentados no Capítulo \ref{ch:machine} utilizando características levantadas no Capítulo \ref{ch:data_preparation}. Essas caracterísiticas evidenciam os bots se considerando a arquitetura das botnets estudada no Capítulo \ref{ch:botnet}. Dessa forma foi desenvolvido uma ferramenta de apoio à decisão na detecção de bots em log DNS a qual as especificações do caso de uso e as janelas de diálogo foram apresentadas no Capítulo \ref{ch:tool}

A primeira contribuição evidente do trabalho foi próprio desenvolvimento da ferramenta de apoio a decisão que reduz o domínio de análise do usuário através dos algoritmos de agrupamento e permite a visualização dos dados seguindo nos eixos as características previamente levantadas como relevante

Além disso, como contribuição de conhecimento pode-se ressaltar a análise da variação da função custo do modelo gerado pelo K-Médias a cada centroide adicionado ao modelo. Percebeu-se que o Método \textit{Elbow} não é satisfatório para essa análise.

Como alternativa para a descoberta da quantidade centroides foi implementada uma solução de visualização dos dados que permite resgatar as informações de cada ponto escolhido. Essa solução ainda não se mostrou a melhor quando aplicada na base de dados conhecida, mas se mostrou muito útil também para detectar pontos que fogem do padrão mais comum.

Foi feita uma análise no log DNS do IME de 12 de março de 2012 e foi alcançado um resultado satisfatório no qual duas das três máquinas suspeitas já conhecidas estavam presentes no menor grupo.

Como já salientado, a complexidade do algoritmo de Agrupamento por Aglomeração é alta, por isso é importante manter um \textit{cache} dos resultados computados pelo algoritmo para que a troca da quantidade de grupos não apresente a demora de uma primeira contrução da hierarquia dos grupos.

Após uma breve comparação dos dados normalizados contra os não-normalizados, percebeu-se o esperado, é melhor trabalhar com dados normalizados. O resultado era esperado pois, algumas características geram números de ordem muito maior do que a maioria e ainda apresentavam grandes diferenças entre si. Para o algoritmo corria-se o risco haver grupos separados por essa distância relativa.

Apesar desse trabalho ter realizado uma análise, o que inclusive foge um pouco do escopo do projeto, espera-se que em trabalhos futuros as análises com outros dados sejam realizadas. Além disso, espera-se um estudo qualitativo das características implementadas um bom trabalho para o log DNS utilizado, mas não necessariamente se comportará bem com todo tipo de fluxo no log DNS. Espera-se também que seja estudado o algoritmo de agrupamento aglomerativo, o qual foi integrado ao trabalho mas não foi teve resultados analisados. Outros algoritmos devem ser analisados, para validar a solução e garantir o pleno apoio à decisão que esse trabalho se propõe.