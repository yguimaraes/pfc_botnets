\chapter{Conclusão}

Nesse trabalho foi descrita a confecção de uma ferramenta de apoio a decisão para detecção de botnets. Primeiramente foi necessário entender o comportamento dos \textit{bots} para então, como realizado logo em seguida, levantar as características do seu comportamento em um Log DNS. Além disso, foram também estudadas as técnicas de agrupamento: algoritmos, técnicas de seleção de parâmetros e técnicas de inicialização de modelo. De posse das informações do problema a ser resolvido e das técnicas de resolução, foi possível modelar a hierarquia de classes e modelo de banco de dados para a confecção do sistema. Foi implementado então um conjunto de executáveis que em um segundo momento foram integrados por uma interface gráfica.

A análise dos dados partiu da premissa de que se deve priorizar modelos simples antes, aumentando a complexidade gradualmente. A ferramenta \textit{Scikit-Learn} que foi utilizada, já dispõe de diversos outros algoritmos de agrupamento implementados, mas que não foram testados por agora, devido aos bons resultados encontrados utilizando um algoritmo mais simples.

O objetivo é oferecer uma detecção razoável que possibilite o usuário a detecção de botnets com comportamentos suspeitos. Isso foi alcançado para o log DNS do IME de 12 de março de 2012, como descrito na seção \ref{sec:db_analysis}, no qual encontrou-se um grupo de 26 clientes dos quais 2 haviam sido identificados como suspeitos previamente. Essa redução do número de elementos é importante para a viabilizar a investigação. Além de fornecer uma ferramenta que auxilie na detecção mesmo de novas botnets que possam surgir, já que a ferramenta não é particular para um tipo de comportamento, apenas reconhece padrões raros, que tendem a ser suspeitos na rede.

Os resultados alcançados com o algoritmo K-médias foram satisfatórios, já que mostrou que o sistema foi capaz de agrupar dois das três máquinas já confirmadas como maliciosas, no menor grupo contendo 26 máquinas. Além disso, a ferramenta visualização, que inicialmente foi proposta como uma forma para analisar as características foi estendida para o auxílio da detecção explicitando o IP e as informações de pontos selecionados pelo usuário.

Outras análises poderiam ser feitas em cima das características levantadas, como por exemplo, o uso de modelos probabilísticos para descriminar pontos de região de baixa probabilidade como suspeitos. Além disso, o trabalho foi realizado com uma quantidade de informação reduzida. É possível que se alcance melhores resultados com a resposta do Servidor DNS à requisição como feito por \citet{schonewille2006domain}, além disso pode-se também considerar outras informações de tráfego na rede.	