\chapter{Conclusão}
O estudo sobre botnets permitiu entender como elas funcionam, entendendo a relevância e a dificuldade do combate desse tipo de ameaça, ao longo desse estudo pôde ser visto também a motivação para a seleção das características que foram posteriormente levantadas.

O objetivo de utilizar o log do DNS não é alcançar a certeza de que todos os bots de uma rede foram encontrados, ou ainda, afirmar com certeza que determinado suspeito é um bot. Mas sim, conseguir oferecer uma detecção razoável, que permita filtrar tráfegos que são legítimos, permitindo que máquinas suspeitas sejam investigadas mais profundamente. Além de fornecer uma ferramenta que auxilie na detecção mesmo de novas botnets que possam surgir, já que a ferramenta não passou por treinamento, apenas reconhece padrões raros, que tendem a ser suspeitos na rede.

A análise dos dados partiu da premissa de que se deve priorizar modelos simples antes, aumentando a complexidade gradualmente. A ferramenta \textit{Scikit-Learn} que foi utilizada, já dispõe de diversos outros algoritmos de agrupamento implementados, mas que não foram testados por agora, devido aos bons resultados encontrados utilizando um algoritmo mais simples. Os resultados alcançados com o algoritmo K-médias foram satisfatórios, já que mostrou que o sistema foi capaz de agrupar dois das três máquinas já confirmadas como maliciosas, no menor grupo contendo 26 máquinas.

Por outro lado o critério de escolha do número de centroides, que também foi escolhido pela sua simplicidade, não foi explícito o suficiente pra decisão. Isso não foi proibitivo para a continuidade do experimento. Porém, pode ser o caso de outro número de centroides melhorar a análise ao utilizar um método com uma metodologia mais rígida para esse cálculo.

Dessa forma, pretende-se direcionar o trabalho agora para implementar uma metodologia já estabelecida para auxiliar na seleção das características mais úteis. Além disso, serão analisados os resultados para os diversos dias para os quais se possui informações sobre o log, com o objetivo de verificar se o bom desempenho dos grupos obtidos será repetido. Finalmente, mas não menos importante, será refinada a saída do sistema, fornecendo mais informações para que os usuários possam identificar com melhor propriedade os motivos que levaram ao sistema acusar as máquinas como suspeitas.