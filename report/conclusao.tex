\chapter{Conclusão}
Após os estudos apresentados, concluiu-se que pela raridade de ocorrência e tendência anomalia nos comportamentos, o melhor tipo de modelo seria Detecção de Anomalia. Atualmente há base de dados IPs infectados etiquetados manualmente, porém é possível que se migre para o cenário Semi-Supervisionado, se for provado que há necessidade de mais dados. A aquisição de amostras é facilitada para este trabalho, pois dispõe-se um servidor de DNS no Instituto que pode fornecer dados.

Além disso dos dados, não há garantias de que o melhor modelo de distribuição de probabilidade poderia ser o Gaussiano. Porém é o escolhido por ser o mais simples. Alternativo a ele, pode-se usar o modelo de Mixturas de Gaussianas, caso existam regiões não contíguas de máquinas em operação normal.

A escolha dos modelos de distribuição de probabilidade, a validação dos características levantas e a constatação da necessidade de mais amostras fazem parte dos trabalhos futuros.