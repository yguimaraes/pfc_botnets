%%
%
% ARQUIVO: dados-pfc.tex
%
% VERSÃO: 1.0
% DATA: Maio de 2016
% AUTOR: Coordenação de Trabalhos Especiais SE/8
% 
%  Arquivo tex com os dados acerca do documento de PFC e da apresentação.
%
%   nos campos que definem nomes (autor; orientador; co-orientador; membros da banca)
%   É PRECISO usar os COMANDOS LaTeX para acentuação, conforme abaixo:
%
%         \'a - á || \`a - à || \~a - ã || \^a - â 
%         \'e - é || \^e - ê || \'i - í 
%         \'o - ó || \~o - õ || \^o - ô 
%         \'u - ú || \"u - ü
%
%%

%%% AUTORES DO PFC (Nome completo)
% ---
%  aceita até 03 autores (de autorI até autorIII)
%    a. preencher sucessivamente a partir de autorI
%    b. REMOVER as definições não necessárias
% ---
\autorI{Jonas Rocha Lima Amaro}
\autorII{Yago Guimar\~aes Coimbra}
%\autorIII{Nome Completo do Terceiro Autor}

%%% POSTOS DOS AUTORES DO PFC
% ---
%  aceita os postos de até 03 autores (de postoautorI até postoautorIII)
%    a. preencher sucessivamente a partir de postoautorI (que deve ser o posto de autorI)
%    b. se o autorX É CIVIL, NÃO DEFINIR postoautorX (remover a linha de definição)
%    c. se o autorX É MILITAR, DEFINIR postoautorX com UMA das seguintes ALTERNATIVAS: Alu / 1 Ten / Cap
% ---
%\postoautorI{1 Ten}
%\postoautorII{Alu}
%\postoautorIII{Cap}

%%% TITULO DO PFC
\titulo{Ferramenta para Detecção de Padrões de Botnet Baseado em Algoritmos de Agrupamento de Aprendizado de Máquina}

%%% DATA DA APRESENTAÇÃO (formato {dd}{Mmmmm}{aaaa})
\datadefesa{27}{Setembro}{2016}

%%% ORIENTADOR DO PFC
% ---
%  CAMPO 1: P (para Prof.); PA (para Profa.); ou qualquer coisa (inclusive VAZIO) - o que for escrito aparecerá no documento
%  CAMPO 2: Nome completo
%  CAMPO 3: D (para D.Sc.); P (para Ph.D.); M (para M.Sc.) ou qualquer coisa (inclusive VAZIO) - o que for escrito aparecerá no documento
%  CAMPO 4: Instituição (com "do / da")
% ---
\orientador{P}{Sergio dos Santos Cardoso Silva}{D}{do IME}

%%% CO-ORIENTADOR DO PFC
% ---
%  se não houver co-orientador, REMOVA ESTA LINHA
%  preenchimento idêntico a \orientador{}{}{}{}
% ---
%\coorientador{P}{Nome Completo do Co-orientador}{P}{do IME}

%%% NÚMERO DA ENTRADA DA BIBLIOTECA (pegar na Biblioteca do IME)
\biblioref{004.69}{S586e}

%%% PALAVRAS-CHAVES DO PFC
% ---
%  devem ser separadas por vírgula e É OBRIGATÓRIO ter pelo menos uma
% ---
\palavraschaves{Botnets, Bots, Detecção de Botnets, Aprendizado de Máquinas, Agrupamento, Detecção de Anomalia}

%%% OUTROS MEMBROS DA BANCA DO PFC
% ---
%  aceita até mais 05 membros (de membrobancaI até membrobancaV)
%    a. preencher sucessivamente a partir de membrobancaI
%    b. REMOVER as definições não necessárias
%
%  cada membro tem preenchimento idêntico a \orientador{}{}{}{}
% ---
\membrobancaI{PA}{Raquel Coelho Gomes Pinto}{D}{do IME}
\membrobancaII{P}{Julio Cesar Duarte}{D}{do IME}
%\membrobancaIII{}{Nome do Membro da Banca 3}{}{da COPPE/UFRJ}
%\membrobancaIV{}{Nome do Membro da Banca 4}{}{da UNIRIO}
%\membrobancaV{}{Nome do Membro da Banca 5}{}{da UERJ}
