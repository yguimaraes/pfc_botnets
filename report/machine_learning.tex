\chapter{Aprendizado de Máquina}
Como \citet{bishop2006pattern} descreve, aprendizado de máquina é uma maneira de abordar um problema de computação. Nessa abordagem, a partir de um grande conjunto de dados, chamados como conjunto de treinamento, são inferidos um conjunto de parâmetros a serem utilizados em um modelo parametrizado.

\section{Definições}

Algumas breves definições serão apresentadas para fim de ambientar o leitor nos temas discutidos neste capítulo.

\begin{description}
\item \textbf{Conjunto de exemplos}: Seja o conjunto de exemplos os dados conhecidos do problema.

\item \textbf{Características}: Sejam características um conjunto ordenado de valores que descrevem um exemplo, a ser modelada pelo algoritmo de Aprendizado.

\item \textbf{Etiquetas}: Seja etiqueta de exemplo, ou simplesmente etiqueta, a saída esperada do modelo para aquela instância.

\item \textbf{Acurária}: A proporção de acertos pelo tamanho do conjunto de exemplos.

\item \textbf{Precisão}: A proporção de exemplos inferidos corretamente como verdadeiros pela quantidade de inferidos como verdadeiro.

\item \textbf{Abrangência}: A proporção de exemplos inferidos corretamente como verdadeiros pela quantidade de acertos.

\item \textbf{\textit{F1 score}}: A média harmônica entre a precisão e a abrangência
\[\frac{2 * \mathrm{precisão} * \mathrm{abrangência}}{\mathrm{precisão} + \mathrm{abrangência}}\]

\item \textbf{Matriz \(\mathbf{X}\)}: Seja \(\mathbf{X}\), exemplos de treinamento, uma matriz \(m \times n\), o qual \(m\) é quantidade de instâncias e \(n\) é a quantidade de características. \(\mathbf{X}\) é a representação matemática do Conjunto de exemplos.

\item \textbf{Vetor \(\mathbf{Y}\)}: Seja \(\mathbf{Y}\), conjunto de etiquetas, um vetor de tamanho \(m\), a quantidade de instâncias, \(\mathbf{Y}\) é o conjunto ordenado das respectivas saídas esperadas de cada linha da matriz \(\mathbf{X}\), ou seja as etiquetas.
\end{description}

É importante destacar a relevância dos conceitos de precisão, abrangência e \textit{F1 score} para problemas que possuem poucos exemplos positivos. Essas definições permitem avaliar esses tipo de dados, como discutiremos na sessão de Detecção de Anomalias.

\section{Categorias de Problemas}

Para ser mais específico, é possível classificar os problemas resolvidos pela aprendizado de máquina em cinco categorias \citep{mohri2012foundations}.

\begin{description}
\item \textbf{Classificação}: Decidir a classe de exemplo dadas às suas características, por exemplo decidir qual dígito foi escrito a apartir de uma imagem de dígito escrita a mão.
\item \textbf{Regressão}: Determinar um valor real para cada exemplo, por exemplo o risco de um paciente ter contraído câncer a partir de imagens e resultados de exames.
\item \textbf{Ordenação}: Ordenar os exemplos a partir de algum critério, por exemplo listar produtos por relevância a partir das palavras chaves da busca do usuário.
\item \textbf{Agrupamento}: Particionar os exemplos em regiões homogêneas, por exemplo identificar comunidades dentro de redes sociais massivas.
\item \textbf{Redução de Dimensionalidade}: Representar o conjunto de exemplos com um número reduzido de dimensões, por exemplo comprimindo imagens para processamento de imagens.
\end{description}

Neste projeto o objetivo é identificar grupos com padrão de comportamento semelhantes. A expectativa é que o comportamento dos bots formem grupos divergentes dos usuários legítimos. Todavia, nem toda máquina pertencente a esse grupo divergente está infectada, o que se quer garantir é um número de reduzido, não mais que 100, de máquinas suspeitas para serem analisadas.

\section{Cenários dos Dados}

Categoriza-se \citep{mohri2012foundations} sete cenários para os algoritmos de aprendizado, esse cenários são fortemente influenciados pelas condições dos dados de treinamento.

\begin{description}
\item \textbf{Aprendizado Supervisionado}: O modelo tem acesso a dados com os resultados de saída já esperados, ou etiquetados, como lê-se na literatura. Os problemas mais comuns desse tipo de cenário são classificação, regressão e ordenação.

\item \textbf{Aprendizado Não Supervisionado}: Só se dispõe da conjunto de exemplos para treinamento sem etiquetas. Geralmente é mais utilizado para classificação, agrupamento e redução de dimensionalidade

\item \textbf{Aprendizado Semi-Supervisionado}: Neste cenário, é possível acessar uma conjunto de exemplos sem etiquetas e uma com etiquetas. Esse é o caso de problemas em que dados sem etiquetas são fáceis de serem adiquiridos, ao contrário dos dados etiquetados, pela dificuldade de etiquetar.

\item \textbf{Inferência Transdutiva}: Semelhante ao Aprendizado Semi-Supervisionado, nem todos os exemplos são etiquetados, mas o modelo só deve ser generalizado apenas para os exemplos conhecidos.

\item \textbf{Aprendizado On-line}: Neste cenário, iterasse no modelo a cada exemplo recebido em rodadas. No início da rodada, o modelo recebe um exemplo, inicialmente sem etiqueta, realiza uma predição, recebe a etiqueta e atualiza os parâmetros do modelo.

\item \textbf{Aprendizado por Reforço}: Neste cenário, os modelos são criados baseado num sistema de recompensa. O modelo é recompensado a cada decisão bem-feita ao interagir com um ambiente.

\item \textbf{Aprendizado Ativo}: O algoritmo é quem realiza requisições a uma entidade capaz de etiquetar exemplos para melhorar os parâmetros do modelo. O objetivo é conseguir gerar um modelo tão bom quando o modelo supervisionado, porém com menos exemplos.

\end{description}

E ainda há outros possíveis cenários ainda mais complexos e específicos. Esses cenários não foram catalogados neste trabalho porque a campo de Aprendizado de Máquina está ainda em constante fase de crescimento.














