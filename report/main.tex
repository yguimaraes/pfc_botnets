%%
%
% ARQUIVO: main.tex
%
% VERSÃO: 1.0
% DATA: Maio de 2016
% AUTOR: Coordenação de Trabalhos Especiais SE/8
% 
%  Arquivo tex principal do documento de Projeto de Fim de Curso (PFC).
%  Este arquivo SÓ PRECISA SER MODIFICADO NA PARTE DE CONTEÚDO:
%
%    a. colocar um \include{•} para cada capítulo do documento de PFC.
%
%%

% -----
% CLASSE DO DOCUMENTO DE PFC
% -----
\documentclass{pfc}

% -----
% PACOTES LATEX USADOS NO DOCUMENTO DE PFC
% -----
\usepackage[brazilian]{babel}
\usepackage[utf8]{inputenc}
\usepackage[T1]{fontenc}

\usepackage{amsmath}
\usepackage{graphicx}
\usepackage{tabularx}
\usepackage{float}
\usepackage{color}
\usepackage{amsfonts,amssymb}
\usepackage[authoryear]{natbib}

\usepackage{enumitem}
\usepackage{rotating}
\usepackage{lipsum}
\usepackage{lastpage}
\usepackage{stringstrings}
\usepackage{pgffor}

\usepackage{pgfgantt} %for tikzpicture
\usepackage{xfrac} % for maths
\usepackage{lscape} % enable landscape page

% -----
% MARGENS DO DOCUMENTO DE PFC
% -----
\usepackage{geometry}
\geometry{
	a4paper,
	total={210mm,297mm},
	left=25mm,
	right=25mm,
	top=25mm,
	bottom=30mm,
	textwidth=160mm,
	textheight=242mm,
	headheight=0mm,
	headsep=0mm,
}

% -----
% DECLARAÇÕES AUXILIARES PARA REFERÊNCIAS
%
%  Diferencia \citet e \citep de acordo com a NBR 10520:2002
% -----
\DeclareRobustCommand{\NATand}{;}
\DeclareRobustCommand{\NATetal}{et~al.}
\makeatletter
\renewcommand{\NAT@nmfmt}[1]{%
  \ifNAT@swa\expandafter\MakeUppercase
  \else\DeclareRobustCommand{\NATand}{ e}\expandafter\@firstofone\fi{{\NAT@up #1}}%
}
\makeatother

% -----
% AMBIENTE DE FIGURAS DE PFC
%
%  A classe do documento está configurada SOMENTE para figuras no formato EPS.
%  Logo, use PREFERENCIALMENTE este tipo de arquivo.
%
%    a. os arquivos das figuras devem estar no diretório 'img'
% -----
\graphicspath{{./img/}}

% -----
% INÍCIO DO DOCUMENTO DE PFC
% -----
\begin{document}

% -----
% PARTE PRÉ-TEXTUAL DE PFC
%
% Alterar o CONTEÚDO dos arquivos siglas.tex E pre-texto.tex
% -----
%%
%
% ARQUIVO: dados-pfc.tex
%
% VERSÃO: 1.0
% DATA: Maio de 2016
% AUTOR: Coordenação de Trabalhos Especiais SE/8
% 
%  Arquivo tex com os dados acerca do documento de PFC e da apresentação.
%
%   nos campos que definem nomes (autor; orientador; co-orientador; membros da banca)
%   É PRECISO usar os COMANDOS LaTeX para acentuação, conforme abaixo:
%
%         \'a - á || \`a - à || \~a - ã || \^a - â 
%         \'e - é || \^e - ê || \'i - í 
%         \'o - ó || \~o - õ || \^o - ô 
%         \'u - ú || \"u - ü
%
%%

%%% AUTORES DO PFC (Nome completo)
% ---
%  aceita até 03 autores (de autorI até autorIII)
%    a. preencher sucessivamente a partir de autorI
%    b. REMOVER as definições não necessárias
% ---
\autorI{Jonas Rocha Lima Amaro}
\autorII{Yago Guimar\~aes Coimbra}
%\autorIII{Nome Completo do Terceiro Autor}

%%% POSTOS DOS AUTORES DO PFC
% ---
%  aceita os postos de até 03 autores (de postoautorI até postoautorIII)
%    a. preencher sucessivamente a partir de postoautorI (que deve ser o posto de autorI)
%    b. se o autorX É CIVIL, NÃO DEFINIR postoautorX (remover a linha de definição)
%    c. se o autorX É MILITAR, DEFINIR postoautorX com UMA das seguintes ALTERNATIVAS: Alu / 1 Ten / Cap
% ---
%\postoautorI{1 Ten}
%\postoautorII{Alu}
%\postoautorIII{Cap}

%%% TITULO DO PFC
\titulo{Ferramenta para Detecção de Padrões de Botnet Baseado em Algoritmos de Agrupamento de Aprendizado de Máquina}

%%% DATA DA APRESENTAÇÃO (formato {dd}{Mmmmm}{aaaa})
\datadefesa{27}{Setembro}{2016}

%%% ORIENTADOR DO PFC
% ---
%  CAMPO 1: P (para Prof.); PA (para Profa.); ou qualquer coisa (inclusive VAZIO) - o que for escrito aparecerá no documento
%  CAMPO 2: Nome completo
%  CAMPO 3: D (para D.Sc.); P (para Ph.D.); M (para M.Sc.) ou qualquer coisa (inclusive VAZIO) - o que for escrito aparecerá no documento
%  CAMPO 4: Instituição (com "do / da")
% ---
\orientador{P}{Sergio dos Santos Cardoso Silva}{D}{do IME}

%%% CO-ORIENTADOR DO PFC
% ---
%  se não houver co-orientador, REMOVA ESTA LINHA
%  preenchimento idêntico a \orientador{}{}{}{}
% ---
%\coorientador{P}{Nome Completo do Co-orientador}{P}{do IME}

%%% NÚMERO DA ENTRADA DA BIBLIOTECA (pegar na Biblioteca do IME)
\biblioref{004.69}{S586e}

%%% PALAVRAS-CHAVES DO PFC
% ---
%  devem ser separadas por vírgula e É OBRIGATÓRIO ter pelo menos uma
% ---
\palavraschaves{Botnets, Bots, Detecção de Botnets, Aprendizado de Máquina, Agrupamento, Detecção de Anomalia}

%%% OUTROS MEMBROS DA BANCA DO PFC
% ---
%  aceita até mais 05 membros (de membrobancaI até membrobancaV)
%    a. preencher sucessivamente a partir de membrobancaI
%    b. REMOVER as definições não necessárias
%
%  cada membro tem preenchimento idêntico a \orientador{}{}{}{}
% ---
\membrobancaI{PA}{Raquel Coelho Gomes Pinto}{D}{do IME}
\membrobancaII{P}{Julio Cesar Duarte}{D}{do IME}
%\membrobancaIII{}{Nome do Membro da Banca 3}{}{da COPPE/UFRJ}
%\membrobancaIV{}{Nome do Membro da Banca 4}{}{da UNIRIO}
%\membrobancaV{}{Nome do Membro da Banca 5}{}{da UERJ}

%%
%
% ARQUIVO: pre-texto.tex
%
% VERSÃO: 1.0
% DATA: Maio de 2016
% AUTOR: Coordenação de Trabalhos Especiais SE/8
% 
%  Arquivo tex para a criação da parte pré-textual do documento de Projeto de Fim de Curso.
%
%%


% -----
% PÁGINA DE CAPA DO DOCUMENTO DE PFC
% -----
\makecapa

% -----
% PÁGINA DE TÍTULO DO PFC
% -----
\prepareadvisors
\maketitle

% -----
% PÁGINA DE CRÉDITOS DO DOCUMENTO DE PFC
% -----
\makecredits

% -----
% PÁGINA DE FOLHA DE ASSINATURAS
% -----
\preparemembers
\approvalpage

% -----
% PÁGINA DE DEDICATÓRIA (OPCIONAL, ie. pode remover toda a página)
% -----
%%% DEDICATÓRIA - PREENCHER...
%\dedicatoria{%
%Ao Instituto Militar de Engenharia, alicerce da minha formação e aperfeiçoamento.
%}%
%\makededication

% -----
% PÁGINA DE AGRADECIMENTOS (OPCIONAL, ie. pode remover toda a página)
% -----
%%% AGRADECIMENTOS - PREENCHER...
%\agradecimentos{%
%Agradeço a todas as pessoas que me incentivaram, apoiaram e possibilitaram esta oportunidade de ampliar meus horizontes. \\
%\indent
%Meus familiares, cônjuge e mestres.\\
%\indent
%Em especial ao meu Professor Orientador Dr. Antonio José Reis e ao Professor Co-orientador Dr. Joel Duarte Silva, por suas disponibilidades e atenções.
%}%
%\makethanks

% -----
% PÁGINA DE EPÍGRAFE (OPCIONAL, ie. pode remover toda a página)
% -----
%%% EPÍGRAFE - PREENCHER...
%\epigrafe{%
%Sem publicação, a ciência é morta.
%}%
%\autorepigrafe{%    %% Se não tem autor, coloque "Anônimo"
%Gerard Piel
%}%
%\makeepigraph

% -----
% PÁGINA DE SUMÁRIO
% -----
\tableofcontents

% -----
% PÁGINAS DE LISTAS DE FIGURAS E DE TABELAS
% se o documento de PFC não possui figuras e/ou tabelas, REMOVA O COMANDO CORRESPONDENTE
% -----
\listoffigures
%\listoftables

% -----
% PÁGINA DE LISTA DE SIGLAS
% se o documento de PFC não possui siglas, REMOVA TODA A PÁGINA
% -----
%%% SIGLAS - PREENCHER...
\acronimo{C\&C}{Comando e Controle}
\acronimo{IRC}{Internet Relay Chat}
\acronimo{HTTP}{HyperText Transfer Protocol}
\acronimo{P2P}{Peer-to-peer}
\acronimo{IDS}{Intrusion Detection System}
\acronimo{DNS}{Domain Name System}
\acronimo{CDCiber}{Centro de Cibernética}

\listofnicks

% -----
% PÁGINA DE LISTA DE ABREVIATURAS
% se o documento de PFC não possui abreviaturas ou símbolos, REMOVA TODA A PÁGINA
% -----
%%% ABREVIATURAS - PREENCHER...
%\abreviatura{Ja}{jacobiano}
%\abreviatura{JS}{fluxo secundário (difusivo)}
%\abreviatura{M}{número de Mach}

%%% SÍMBOLOS - PREENCHER...
%\simbolo{$\Phi$}{termo de dissipação viscosa}
%\simbolo{$\Gamma$}{coeficiente de difusão efetivo}
%\simbolo{$\alpha$}{fator de sub-relaxação}
%\simbolo{$\phi$}{variável dependente da equação diferencial geral}

%\listofsymbols

% -----
% PÁGINA DE RESUMO
% -----
%%% RESUMO - PREENCHER...
\resumo{%
Botnets são uma ameaça cibernética que já trouxe muito prejuízo \citep{silva2013botnets}. Essa ameaça utiliza computadores infectados para realizar atividades fraudulentas como servir páginas piratas para roubar informações sensíveis, enviar \textit{spam} para usuários comuns e enviar sucessivas requisições para derrubar servidores. Por ser uma atividade ilegal, os criminosos realizam a comunicação entre as máquinas com comportamentos divergentes.\\
\indent
Baseado nessa premissa, esse projeto se propõe a auxiliar na detecção de máquinas que pertencem à botnets utilizando algoritmos de Aprendizado de Máquina baseado em Agrupamento.\\
\indent
Dentro do contexto Exército Brasileiro, o projeto auxiliará a Inteligência do Exército Brasileiro na prevenção de ataques por botnets. Esse sistema aspira fazer parte do raciocinador para um sistema que desarticula botnets.
}%
\makeresumo

% -----
% PÁGINA DE ABSTRACT
% -----
%%% ABSTRACT - PREENCHER...
\abstract{%
Botnets are a cyber threat that already brought plenty of money dispend \citep{silva2013botnets}. This threat uses infected computers to perform fraudulent activities, such as serving pirated sites to steal sensible information, sending spam to common users and sending successive requests to get servers down. Because it is an illegal activity, the criminals do the communication between the machines with divergent behaviours.\\
\indent
Based on that premise, this project proposes to detect machines that are part of a botnet using Clustering flavored Machine Learning algorithms.\\
\indent
Inside the Brazilian Army context, the project will help the Inteligence of Brazilian Army preventing botnet attacks. This system aspires to be an important piece of an integrated botnet desarticulating system.
}%
\makeabstract


\parindent 0.75cm

% -----
% PARTE DE CONTEÚDO DE PFC
%
%  Escrever cada capitulo do documento de PFC em um arquivo .tex separado.
%  Adicionar os arquivos .tex ao documento com comando \include{•}
% -----

\chapter{Introdução}
O espaço cibernético sofre com um crescente número de ataques maliciosos. Esse fato fez com que a pesquisa na área de defesa cibernética ganhasse muita importância. Ademais, novas formas de ataques cibernéticos são criadas e antigas são atualizadas, dificultando o ideal de livrar nossas máquinas de qualquer ameaça \citep{bharathula2016equitable}. Uma dessas formas de ameaças, responsável por grande parte dos ataques em larga escala pela Internet atualmente, é formada pelas botnets.

As botnets, que são redes formadas por máquinas infectadas por alguma forma de \textit{malware}, apresentaram um elevado crescimento no início do século XXI, se tornando uma das ameaças mais desafiadoras no campo de defesa cibernética \citep{chang2015measuring}. A detecção dessas redes se tornou um tópico muito importante na área de defesa cibernética, devido ao desafio que é realizar essa detecção, pois, as botnets são muito flexíveis e robustas, além de estarem em processo de aprimoramento contínuo \citep{bu2010new}.

Com a necessidade de realizar essa detecção, mesmo com as mudanças realizadas no funcionamento das botnets ao longo do tempo, o emprego de técnicas de aprendizado de máquina é muito promissor, especialmente as técnicas de agrupamento. É esperado que os bots que formam as botnets, tenham comportamento diferente de usuários convencionais, e, dessa forma, espera-se que eles sejam agrupados em grupos menores dos formados pela massa de usuários normais.

Levando em conta essas características, o presente trabalho busca desenvolver uma ferramenta para auxiliar na detecção de botnets, utilizando as informações das consultas ao servidor DNS realizadas na rede através de algoritmos de aprendizado de máquina.

\section{Objetivo}
O objetivo deste trabalho é desenvolver uma ferramenta para auxiliar na detecção de possíveis hospedeiros de bots em uma botnet, reduzindo o trabalho humano utilizado na detecção das botnets. Isso será feito utilizando algoritmos de agrupamento, que utilizarão dados que serão calculados através das informações contidas em consultas de DNS realizadas. 

Além disso, o desempenho dessa ferramenta será avaliado, utilizando as informações do log de DNS dos servidores do IME, para o qual já temos algumas máquinas infectadas previamente mapeadas por inspeção e investigação manuais.

\section{Motivação}
O crescimento e diversificação do uso da Internet, criaram o cenário ideal para o desenvolvimento das botnets, que são consideradas uma das maiores ameaças atuais na área de segurança da informação \citep{ji2008botnet}. 

Esta ameaça requer uma atenção especial, principalmente no âmbito militar, devido a aplicação que as botnets podem ter no contexto de guerra cibernética. Permitindo que os atacantes inutilizem servidores do inimigo e/ou coletem informações confidenciais.

A detecção das botnets é um grande desafio, pois os atacantes estão sempre aprimorando o funcionamento dos bots, com isso os mecanismos atuais muitas vezes falham ao detectar novas implementações de botnets. Isso motivou o uso de algoritmos de agrupamento para serem utilizados na detecção de botnets, de forma que a ferramenta seja capaz de detectar inclusive botnets que eram desconhecidas.

Por esses motivos, torna-se muito clara a necessidade da criação de uma ferramenta que permita facilitar o trabalho de detecção de novas botnets, não previstas pelas ferramentas atuais. Já que novas botnets só poderiam ser identificadas por inspeção manual, que é um processo bastante complexo devido à enorme quantidade de informações presentes em uma rede.

\section{Justificativa}
A justificativa desse trabalho é contribuir para a área de defessa cibernética, uma área de interesse do IME, apoiando à pesquisa com o conhecimento de características de consultas de DNS que podem ser relevantes para realizar a detecção de botnets. Além disso, a ferramenta pode ser utilizada pelo Centro de Defesa Cibernética (CDCiber) ou servir como um módulo de um sistema integrado de detecção de botnets \citep{silva2012arquitetura}.

\section{Metodologia}
Este trabalho foi dividido em cinco fases principais: estudo teórico, preparação dos dados, implementação dos algoritmos de agrupamento aos dados preparados, breve análise e testes dos resultados e implementação da interface gráfica. 

Na etapa de estudo teórico foi feito um levantamento da funcionalidade das botnets e do funcionamento e aplicações dos algoritmos de aprendizado de máquina, com o objetivo de identificar vulnerabilidades e padrões nas botnets, motivando a seleção de características relevantes dos dados utilizados, de forma a melhorar a performance do algoritmo.

Na etapa seguinte, foi feito um levantamento das características candidatas a serem utilizadas pelo algoritmo da ferramenta. De posse dessas características, foi implementado um programa em C++ para realizar a extração automática dessas características da base de dados que utilizamos.

Na etapa de implementação dos algoritmos de agrupamento aos dados preparados, utilizamos a ferramenta \textit{Scikit-Learn} para desenvolvermos um programa escrito em Python para aplicar algoritmos de aprendizado de máquina nas características previamente levantadas.

Posteriomente, foi feita uma breve análise, para identificar as técnicas mais adequadas, possíveis refinamentos na ferramenta e comprovar sua eficiência na detecção de bots.

Por fim, foi implementada uma interface gráfica para facilitar o uso da ferramenta pelos usuários, integrando todos as funcionalidades desenvolvidas em uma aplicação multi-plataforma utilizando a ferramenta \textit{Qt Creator}.

\section{Estrutura}
Este trabalho é composto por sete capítulos, iniciando com esta introdução. No capítulo 2 é feito um estudo teórico sobre botnets para identificar o tipo de \textit{malware} que desejamos identificar. Em seguida, no capítulo 3 é realizado um estudo sobre algoritmos de aprendizado de máquina dando ênfase nos algoritmos de agrupamento, que será a estratégia utilizada neste trabalho. Em seguida, no capítulo 4 são explicados os procedimentos e a motivação que levaram à forma de tratamento dos dados. O capítulo 5, mostra quais foram os resultados preliminares encontrados, fazendo uma análise sobre as características dos grupos formados, mostrando as motivações para os próximos aprimoramentos. No capítulo 6, são mostradas as funcionalidades do sistema desenvolvido com a interface gráfica e como foi feita sua arquitetura. Ao final, no capítulo 7, estão as conclusões que foram obtidas.
\chapter{Botnets}
As Botnets são redes formadas por máquinas infectadas com \textit{malware}, permitindo que o atacante (\textit{botmaster}) realize diversas atividades criminais remotamente, como roubo de informações, ataques de negação de serviço, envio de SPAM, etc. \citep{silva2013botnets}.

Com o crescimento e diversificação do uso da Internet, o meio cibernético se tornou mais relevante e mais atraente para a realização de ataques maliciosos. Isso motivou o crescimento do número de botnets existentes e aumentou o potencial de contaminação das mesmas, além disso, para evitar os mecanismos de detecção existentes, elas se tornaram cada vez mais sofisticadas.

Para que o detector se torne mais robusto, é preciso compreender o funcionamento das botnets e seus objetivos. Esse conhecimento é necessário para entender as configurações existentes nas botnets atuais, além de compreender como essas configurações podem evoluir. De posse desse conhecimento, espera-se que seja possível identificar características relevantes e intrínsecas ao funcionamento das botnets, mesmo quando o \textit{botmaster} estiver tentando evitar os mecanismos de detecção.

\section{Elementos das Botnets}
As botnets apresentam alguns elementos estruturais tipicamente envolvidos, que estão presentes independente do protocolo ou arquitetura utilizada. A Figura \ref{fig:typical_elements} mostra a estrutura desses elementos e como eles se relacionam em uma botnet. Segue uma descrição para cada componente:
\begin{itemize}  
\item Bots: São \textit{malwares} instalados nos computadores das vítimas que podem realizar as ações maliciosas que o \textit{botmaster} envia através do canal de comando e controle (C\&C). Geralmente, o \textit{malware} é inicializado quando o hospedeiro inicializa a máquina, porém isso pode ser configurado pelo \textit{botmaster} para dificultar a detecção da atividade maliciosa.
\item Hospedeiros: São as máquinas em que o bot foi instalado, ou seja, infectadas \citep{puri2003bots}.
\item \textit{Botmaster}: é o indivíduo que configura o bot, dissemina e controla a botnet.
\item Canal de Comando e Controle (C\&C): é o meio que o \textit{botmaster} tem para se comunicar com a sua botnet. É a parte chave do funcionamento, pois é necessário para o envio dos comandos de atividade maliciosa aos hospedeiros. Dessa forma, grande parte das características da botnet, como robustez, facilidade de detecção/desativação, estabilidade, etc., são definidas pela forma que a infraestrutura de C\&C está organizada.
\end{itemize}

\begin{figure}
\includegraphics[width=\textwidth]{typical_elements}
\caption[Elementos das botnets]{Elementos das botnets \citep{silva2013botnets}} \label{fig:typical_elements}
\end{figure}

\section{Ameaças e Formas de Defesa}
O crescimento do número de máquinas conectadas constantemente a enlaces de alta velocidade e rodando sistemas com vulnerabilidades consideráveis, criou um ambiente favorável à formação de botnets. Além disso, muitas vezes o bot é transparente ao responsável pela máquina infectada, ou seja, não atrapalha o funcionamento da máquina, fazendo com que a vítima não perceba a infecção e tente combatê-la. Esses fatores, aliados ao enorme potencial de causar danos, fazem com que as botnets sejam um dos maiores desafios de pesquisa em segurança no espaço cibernético atual \citep{soltani2014survey}.

Existem características que tornam o \textit{host} mais interessante ao \textit{botmaster} como: altas taxas de transmissão, baixos níveis de segurança e monitoração, alta disponibilidade e localização distante (dificultando que as agências reguladoras detectem as atividades, já que os bots estarão espalhados por diversas nações). Esses fatores ajudam o bot a passar desapercebido e a contribuir com maior capacidade de banda ao \textit{botmaster}, facilitando ataques como os de negação de serviço.

Existem duas formas para combater um ataque realizado por botnets: reativamente ou preventivamente. A forma reativa é a mais comum e envolve detectar a existência da atividade maliciosa e reagir ao ataque tentando reduzir o tráfego malicioso para níveis aceitáveis. Uma desvantagem é que o ataque já vai ter sido inicializado quando for detectado, ou seja, já vai haver causado danos antes de ser solucionado. A forma preventiva busca evitar que a botnet possa realizar alguma atividade maliciosa, porém essa atividade não é simples, já que o atacante pode aprimorar seus bots, tornando-os mais sofisticados, exigindo grandes investimentos para manter os recursos de segurança atualizados.

O mecanismo que estamos desenvolvendo é da forma preventiva, já que o algoritmo busca encontrar padrões e identificar possíveis máquinas infectadas por botnets. Tudo isso na fase em que o \textit{botmaster} ainda está infectando máquinas com o bot para a botnet. Ou seja, buscamos detectar as máquinas infectadas antes do ataque, como um ataque de negação de serviço, em si ser efetivado. Uma característica desejável para um detector é a detecção em tempo real, com o objetivo de minimizar os danos causados e o tempo de reação do \textit{botmaster}. Porém, obter essa característica é um desafio, devido ao grande número de dados que devem ser tratados e analisados. Dessa forma, nosso projeto não fará detecção em tempo real, mas tentará se aproximar disso, utilizando a detecção dos dados coletados ao longo de um dia para detectar bots que atuaram nas últimas 24 horas.

\section{Ciclo de Vida das Botnets}
Na maioria dos casos, existe um ciclo com fases bem definidas de como uma botnet é criada e mantida, a Figura \ref{fig:botnets_lifecycle} mostra essas fases para cada novo hospedeiro que é contaminado.

\begin{figure}
\tikzstyle{block} = [rectangle, draw, text width=9em, text centered, rounded corners, minimum height=4em]
\tikzstyle{line} = [draw, -latex']
\centering
\begin{tikzpicture}[node distance = 4.5cm, auto]
    % Place nodes
    \node [block] (init) {Infecção Inicial};
    \node [block, below of=init] (second) {Injeção Secundária};
    \node [block, right of=init] (connection) {Conexão};
    \node [block, below of=connection] (malicious) {Atividades Maliciosas};
    \node [block, right of=malicious] (update) {Manutenção e Atualização};
    % Draw edges
    \path [line] (init) -- (second);
    \path [line] (second) -- (connection);
    \path [line] (connection) |- +(4,2) |- (connection.east); 
    \path [line] (connection) -- (malicious);
    \path [line] (malicious) -- (update);
    \path [line] (update) -- (connection);

\end{tikzpicture}
\caption[Ciclo de Vida das Botnets]{Ciclo de Vida das Botnets} \label{fig:botnets_lifecycle}
\end{figure}

Na primeira fase, chamada de injeção inicial, o atacante procura vulnerabilidades na máquina do futuro hospedeiro para explorá-las e infectá-lo com o \textit{malware}, tornando-se um bot em potencial, isso pode ocorrer, por exemplo, através de um \textit{download} indesejado ou através de um anexo em um e-mail. Após a infecção ser bem sucedida, ocorre a injeção secundária: o host infectado, através do \textit{malware} inicial instalado, busca em uma rede os reais binários do \textit{malware} do bot, os quais após baixados e executados concluirão a infecção e transformam o host em um bot real.\citep{feily2009survey}

Durante a fase de conexão, o bot estabelece conexão com o canal de C\&C. Isso se repete sempre que o host é reiniciado, podendo ser considerada uma fase vulnerável já que é uma fase essencial, além de geralmente seguir um padrão. Após a efetivação da conexão, o bot se torna ativo na botnet, e passa a realizar os comandos enviados pelo \textit{botmaster} através do canal de C\&C, efetivando as atividades maliciosas solicitadas. A última fase é a de manutenção e atualização, e tem por objetivo manter a botnet ativa e atualizada, já que se o \textit{botmaster} deseja que os bots possam evitar novas técnicas de detecção, adicionar novas funcionalidades ou até mesmo alterar o servidor de C\&C, os binários do programa bot devem ser modificados.

\section{Arquitetura das Botnets}
Existem 4 tipos de arquiteturas para as botnets: centralizada, descentralizada, híbrida e aleatória. 

Na arquitetura centralizada, mostrada na Figura \ref{fig:centralized_architecture} todos os bots se comunicam com um número pequeno de servidores de C\&C. Embora ela ofereça vantagens ao \textit{botmaster}, como baixa latência e facilidade de manutenção, ela também torna a botnet bastante vulnerável, permitindo que a botnet seja desligada após a identificação dos poucos pontos centrais de C\&C. Esta arquitetura é muito utilizada por botnets que utilizam o protocolo IRC (\textit{Internet Relay Chat}) para comunicação. Todavia, o tráfego desse protocolo é incomum e raramente utilizado, especialmente em ambientes corporativos, por esse motivo, o tráfego desse tipo de protocolo costuma ser bloqueado, inutilizando a botnet. Devido a isso, o uso do protocolo HTTP (\textit{HyperText Transfer Protocol}) se popularizou já que ele é amplamente utilizado, disfarçando as comunicações das botnets.

\begin{figure}
\includegraphics[width=\textwidth]{centralized}
\caption[Arquitetura Centralizada]{Arquitetura Centralizada\citep{wang2010advanced}} \label{fig:centralized_architecture}
\end{figure}

A fragilidade da arquitetura centralizada, motivou o desenvolvimento da arquitetura descentralizada, na qual uma variedade de protocolos P2P (\textit{Peer-to-peer}) é utilizada. A flexibilidade e robustez dessa arquitetura, permite que mesmo que muitos bots sejam desativados a botnet possa continuar funcionando, já que não existem pontos centralizados de C\&C. 

As arquiteturas híbridas apresentam características de ambas as arquiteturas centralizadas e descentralizadas, como mostrado na Figura \ref{fig:hybrid_architecture}, na qual os bots são classificados em dois grupos: clientes e serventes. Os serventes exercem os papéis tanto de clientes quanto servidores, possuindo endereço de IP estático e público para serem acessíveis globalmente, sendo utilizados para repassar os comandos enviados pelo \textit{botmaster}. Os demais bots, são denominados clientes pois não aceitam comunicações de entrada, dessa forma e podem apresentar IP dinâmico, privado ou protegidos por \textit{firewall} para não serem roteados facilmente. 

Por fim, a arquitetura aleatória é um modelo, até agora, apenas teórico. No qual o bot não se comunica ativamente com o \textit{botmaster} ou com outros bot. Dessa forma, para realizar um ataque, o \textit{botmaster} precisa vasculhar a rede em busca de um bot para enviar o comando e realizar as atividades maliciosas.

\begin{figure}
\includegraphics[width=\textwidth]{hybrid}
\caption[Arquitetura Híbrida]{Arquitetura Híbrida\citep{wang2010advanced}} \label{fig:hybrid_architecture}
\end{figure}

\section{Detecção de Botnets}

Existem duas categorias de técnicas para detecção de botnets: \textit{honeynets} e sistemas de detecção de intrusos (IDS). 

As \textit{honeynets} consistem na criação de redes que apresentam vulnerabilidades com a intenção de que elas sejam comprometidas com \textit{malwares}. Isso permite que informações sobre a botnet sejam captadas e estudadas. Por isso elas são consideradas mais efetivas para compreender as características de uma botnet do que para realizar a detecção de botnets.

A detecção por IDS, pode ser classificada entre duas técnicas: a baseada em assinaturas e a baseada em anomalias. A técnica baseada em assinaturas, consiste em extrair padrões da rede e comparar com um banco de dados onde se encontram os padrões que já foram vistos em botnets, ou seja, ela não permite que novas botnets sejam identificadas e envolve a posse de um banco de dados enorme com o maior número de informações existentes sobre as botnets previamente detectadas. Dessa forma, a técnica baseada em anomalias é a principal área de pesquisa para detecção de botnets, baseando-se em anomalias na rede, como alta latência, aumento no tráfego ou uso de portas incomuns para detectar a presença de bots na rede.

As técnicas baseadas em anomalias, podem ser focadas no host ou na rede. Nas técnicas focadas no host, cada máquina possui uma ferramenta de monitoração instalada (o que não é muito escalável), e tem seu comportamento analisado para verificar a existência de atividades suspeitas. Já as técnicas focadas na rede, podem ser feitas de forma ativa (que possuem a grande desvantagem de aumentar o tráfego da rede ao injetar pacotes com a finalidade de examinar se um cliente é humano ou um bot) ou passivamente, sendo esta última a forma de detecção mais utilizada e pesquisada atualmente.

A monitoração passiva de uma rede consiste em analisar o tráfego da rede buscando por comunicações suspeitas que podem ter sido enviadas pelos bots ou canais de C\&C. Essa monitoração é possível pois os bots de uma mesma botnet costumam apresentar padrões de comunicação, já que eles são pré-programados pelo mesmo \textit{botmaster} para entrar contato com o servidor de C\&C.

Para que a análise do tráfego seja viabilizada, são empregadas diversas técnicas como métodos estatísticos, mineração de tráfego, teoria de grafos, agrupamento, modelos estocásticos, redes neurais, entre outras.

A detecção de botnets é uma tarefa bastante desafiadora porque os \textit{botmasters} estão sempre aprimorando os bots, tornando-os mais difíceis de serem detectados. Por exemplo, as primeiras detecções buscavam mensagens suspeitas nos conteúdos da mensagem, afim de evitar isso os \textit{botmasters} passaram a utilizar criptografia tornando essa técnica de detecção obsoleta. Outra dificuldade para algoritmos de agrupamento é que podem ser evitados usando técnicas de randomização nas comunicações e atribuição de tarefas diferentes para os bots.
\chapter{Aprendizado de Máquina}
Como \citet{bishop2006pattern} descreve, aprendizado de máquina é uma maneira de abordar um problema de computação. Nessa abordagem, a partir de um grande conjunto de dados, chamados conjunto de treinamento, são inferidos um conjunto de parâmetros a serem utilizados em um modelo parametrizado.

\section{Definições}

Algumas breves definições serão apresentadas com a finalidade de ambientar o leitor nos temas discutidos neste capítulo.

\begin{description}
\item \textbf{Conjunto de exemplos}: É o conjunto de entidades modeladas no problema. Essas entidades poderiam ser pessoas, máquinas, imagens, sinais de som, enfim qualquer objeto que se queira estudar.

\item \textbf{Características}: É um conjunto ordenado de valores que descrevem um exemplo. A escolha das características utilizadas pra representar os exemplos traz consequências diretas ao problema.

\item \textbf{Etiqueta}: É a classe que um exemplo particular pertence. Essa definição será utilizada para descrever o diversos tipos de cenário que os dados podem se encontrar.

\item \textbf{Função Custo}: É a função que mede quanto o modelo se adequa ao conjunto de exemplos. Por exemplo, em uma regressão linear utiliza-se a soma das distâncias entre as características dos exemplos e o modelo linear que se quer utilizar para generalizar essas características.
\end{description}


\section{Categorias de Problemas}

Dentro do universo de problemas que são resolvidos pelas técnicas de aprendizado de máquina, é possível classificá-los em cinco categorias \citep{mohri2012foundations}.

\begin{description}
\item \textbf{Classificação}: Decidir a classe de um exemplo partindo das suas características, por exemplo decidir qual dígito foi escrito conhecendo uma imagem de dígito escrita a mão.
\item \textbf{Regressão}: Determinar um valor real para cada exemplo, por exemplo calcular o risco de um paciente ter contraído câncer a partir de imagens e resultados de exames.
\item \textbf{Ordenação}: Ordenar os exemplos a partir de algum critério, por exemplo listar produtos por relevância a partir das palavras chaves da busca do usuário.
\item \textbf{Agrupamento}: Particionar os exemplos em regiões homogêneas, por exemplo identificar comunidades dentro de redes sociais massivas.
\item \textbf{Redução de Dimensionalidade}: Representar o conjunto de exemplos com um número reduzido de dimensões, por exemplo comprimir imagens para processamento de imagens.
\end{description}

Neste projeto o objetivo é identificar grupos com padrão de comportamento semelhantes. A expectativa é que o comportamento dos bots formem grupos divergentes dos usuários legítimos. Todavia, nem toda máquina pertencente a esse grupo divergente está infectada, o que se quer garantir é um número de reduzido, da ordem de \(10\), de máquinas suspeitas para serem analisadas.

\section{Cenários dos Dados}

Categoriza-se \citep{mohri2012foundations} sete cenários para os algoritmos de aprendizado, esse cenários são fortemente influenciados pelas condições dos dados de treinamento.

\begin{description}
\item \textbf{Aprendizado Supervisionado}: O modelo tem acesso a dados com os resultados de saída já esperados, ou etiquetados, como lê-se na literatura. Os problemas mais comuns desse tipo de cenário são classificação, regressão e ordenação.

\item \textbf{Aprendizado Não Supervisionado}: Só se dispõe da conjunto de exemplos para treinamento sem etiquetas. Geralmente é mais utilizado para classificação, agrupamento e redução de dimensionalidade

\item \textbf{Aprendizado Semi-Supervisionado}: Neste cenário, é possível acessar uma conjunto de exemplos sem etiquetas e uma com etiquetas. Esse é o caso de problemas em que dados sem etiquetas são fáceis de serem adquiridos, ao contrário dos dados etiquetados, pela dificuldade de etiquetar.

\end{description}

E ainda há outros possíveis cenários ainda mais complexos e específicos. Esses cenários não foram catalogados neste trabalho porque a campo de Aprendizado de Máquina está ainda em constante fase de crescimento.

No caso deste trabalho, algumas máquinas identificadas pelo IP foram confirmadas como bots, mas não há garantia de que sejam os únicos bots na rede. O algoritmo se vê num cenário não supervisionado, mas os desenvolvedores validam os resultados a partir dos bots já conhecidos.

Note que se as máquinas conhecidamente suspeitas não se encontrarem no grupo divergente, o sistema falhou.

\section{Algoritmos de Agrupamento}

Existem diversas definições pra grupos de entidades similares \citep{faceli2011inteligencia}, dentre elas pode citar:
\begin{description}
\item \textbf{Grupos bem separado}: Cada exemplo de um grupo está mais próximo de outro exemplo do mesmo grupo do que a qualquer outro ponto não pertencente a ele. Essa definição se assemelha bastante à abordagem de aprendizado K vizinhos mais próximos.
\item \textbf{Grupo baseado em centro}: Cada exemplo do grupo está mais próximo do centróide do próprio grupo do que do centróide de qualquer outro grupo. Essa é a premissa básica do K-médias.
\item \textbf{Grupo contínuo}: Qualquer exemplo de um grupos está mais próximo a um mais exemplos do mesmo grupo do que a qualquer outro exemplo não pertencente ao grupo. Essa definição é bastante similar ao \textit{Grupo bem separado}, porém neste basta estar mais próximo a um conjunto reduzido do mesmo grupo, diferente do bem separado que precisa estar mais próximo a todos do mesmo grupo.
\item \textbf{Grupo baseado em densidade}: Um grupo é uma região de alta densidade separada de outras regiões de alta densidade por zonas de baixa densidade.
\item \textbf{Grupo baseado em similaridade}: Um grupo é um conjunto de pontos similares contratados por outros grupos que não são similiares.
\end{description}

Cada definição induz uma abordagem de algoritmo diferente. Assumiu-se o modelo mais simples baseado em centros para resolver o problema, baseado em centros através do algoritmo de K-médias que será apresentado detalhadamente na seção \ref{sec:kmeans}. Porém outros modelos, como o baseado em similaridade, poderiam vir a ser úteis também, pois o objetivo é identificar grupos baseado em comportamentos tipicamente divergentes do comportamento comum.

\section{K-Médias}\label{sec:kmeans}

Segundo \citet{witten2011data}, K-Médias é um algoritmo de aprendizado iterativo baseado na distância geométrica. Inicialmente o algoritmo inicializa \textit{K} centroides em posições distintas. Cada exemplo, então, é associado ao centroide mais próximo. Em seguida o centroide atualiza sua posição para a média das posições dos pontos associados a ele. A partir de então recomeça o ciclo, são reassociados os pontos que estão mais próximos do centroide em sua nova posição.

Note que a função custo que se espera minimizar é a soma de todas as distâncias entre os exemplos e seus respectivos centroides, como descrito na seguinte equação

\[
\sum_{i=1}^{n} \lVert cent(\mathbf{X_{i}}) - \mathbf{X_{i}} \rVert
\]

O qual \(cent(\mathbf{X})\) é a função que retorna o centroide mais próximo das coordenadas \(\mathbf{X}\)

\begin{figure}
\includegraphics[width=\textwidth]{k_means_example}
\caption[Aplicação do K-Médias]{Aplicação do K-Médias} \label{fig:k_means_example}
\end{figure}

A \ref{fig:k_means_example} representa a aplicação da técnica em dados bidimensionais com três grupos. É possível notar que converge rapidamente, nesse caso em apenas 9 iterações chegou-se a um resultado satisfatório. Além disso, espera-se uma quantidade de máquinas da ordem de \(10^4\), como o algoritmo K-Médias é linear tanto na decisão do grupo para cada instância quanto no cálculo do novo centroide, a temporização do algoritmo é muito baixa.

\subsection{Determinação do Número de Grupos}

Para a operação plena do algoritmo é necessário que seja informada a quantidade de grupos que são buscados. Isso pode ser um problema simples quando é possível visualizar os dados, como na figura \ref{fig:k_means_example}, mas não é possível realizar o mesmo procedimento quando os exemplos têm mais que três dimensões, como no caso deste projeto.

\begin{figure}
\includegraphics[width=\textwidth]{between_two_clusters}
\caption[Centróide Pouco Representativo]{Centróide Pouco Representativo} \label{fig:between_two_clusters}
\end{figure}

Além disso, se por um lado, pode-se carecer centroides levando a cenários como na figura \ref{fig:between_two_clusters}, no qual o centroide não generaliza os dados, por outro, como é possível visualizar na figura \ref{fig:cluster_overfitting}, um grupo pode acabar se separando por centroides desnecessários.

\begin{figure}[htbp]
\centering
\includegraphics[scale=0.5]{cluster_overfitting}
\caption[Grupo Dividido]{Grupo Dividido} \label{fig:cluster_overfitting}
\end{figure}

Para resolver esse problema utiliza-se o Método \textit{Elbow} \citep{kodinariya2013review}. Trata-se de um método visual no qual procura-se o ponto em que a adição de um novo centroide não implica mais na redução drástica da função custo. Esse efeito é observado na figura \ref{fig:elbow}

\begin{figure}[htbp]
\centering
\includegraphics[scale=0.5]{elbow}
\caption[Ponto Crítico da Otimização]{Ponto Crítico da Otimização} \label{fig:elbow}
\end{figure}

O artigo \citep{kodinariya2013review} também cita outros métodos para com critérios matemáticos menos subjetivos, porém este foi utilizado para fins de simplicidade. No capítulo 5 relativo aos experimentos, esse método será avaliado para os dados que utilizados neste trabalho.
\chapter{Preparação Dos Dados} \label{ch:data_preparation}
Muitos dos métodos existentes para detecção de botnets, utilizam informações completas do tráfego da rede, inclusive de informações do \textit{payload} para extrair as características relevantes \citep{krmicek2011inspecting}. Infelizmente, nem sempre todas essas informações estão presentes, por diversos motivos, como questões de privacidade ou falta de autorização para acessar o conteúdo dos pacotes. Por isso, torna-se necessário analisar a capacidade de utilizar informações mais simples, como o fluxo da rede (\textit{NetFlow}) ou o registro do log de requisições a um servidor DNS, como características.

Nesse trabalho vamos focar nos dados das requisições coletadas pelo servidor DNS do IME no período de fevereiro a abril de 2012. Ao longo desse capítulo mostraremos quais as informações contidas nesses registros e as características que consideramos mais relevantes para auxiliar na detecção de botnets.

\section{Estrutura do Log Bruto de DNS }
O log de DNS utilizado foi baseado nos dados coletados pelo servidor DNS do IME. Esses dados são privados e foram coletados em diversos dias de fevereiro a abril de 2012. Cada registro no DNS é uma requisição que foi feita ao servidor por uma máquina cliente. As informações contidas em cada registro são:

\begin{itemize}
\item Data em que a requisição foi feita
\item Horário com precisão de milésimos da requisição
\item Endereço IP da máquina que fez a requisição
\item Porta do Cliente
\item Nome do domínio requisitado
\item Tipo de Requisição
\item IP do Servidor DNS consultado
\end{itemize}

Abaixo, encontra-se um exemplo de uma entrada no registro de requisições extraído da base de dados: 

\begin{quote}
11-Mar-2012 12:24:16.772 queries: info: client 41.128.225.42\#57135: query: rEcREIo.DE9.iMe.eb.br IN A - (200.20.120.33)
\end{quote}

Dado que os dados não estão estruturados em um formato facilmente reconhecido pelo computador, como em um arquivo JSON ou XML, foi preciso realizar a extração dos dados do log utilizando expressões regulares. A expressão regular foi construída inicialmente para reconhecer alguns exemplos de entradas. Após isso, ela foi aperfeiçoada para reconhecer novos exemplos ao colocar um mecanismo para que o sistema avise quando a entrada não foi reconhecida pela expressão regular. Dessa forma, a expressão era adaptada para reconhecer entradas válidas e que não forem reconhecidas. O mecanismo de aviso foi mantido, afim de garantir o aviso ao usuário de entradas no log que não foram previstas pela expressão regular desenvolvida.

Após aplicar a expressão regular em cada linha extraída do log do servidor DNS do IME, foram filtradas as requisições que foram feitas por máquinas que pertencem à infraestrutura do IME, como o servidor de correio, já que são seguras e muitas vezes responsáveis por mais de 50\% das entradas no log nos dias analisados. Depois de aplicar esse filtro, o dado tratado pode ser armazenado de modo amigável para a criação do modelo.

\section{Levantamento das Características}
\label{sec:levantamento_das_caracteristicas}
De posse das informações apresentadas nos capítulos anteriores, é possível definir mais claramente o problema que esse trabalho se propõe a resolver e, em linhas gerais, como ele será abordado.

O objetivo desse trabalho é a partir das informações obtidas no log de registros de um servidor DNS, identificar quais máquinas na rede são suspeitas de pertencer a uma botnet e merecem atenção para uma investigação mais profunda. Para isso, foram determinadas características que uma máquina pertencente a uma botnet pode exibir que a distingue de uma máquina com uso normal.

As características analisadas foram divididas em quatro tipos, que identificam comportamentos divergentes do uso comum que esperam ser observados: a forma de escolha dos domínios, o comportamento de máquina, os domínios visitados em comum e tipos de requisição.

\subsection{Escolha dos Domínios}
Muitas botnets utilizam técnicas de geração de domínio \citep{zhou2013dga}. Essa técnica permite que os bots consultem um grande conjunto de domínios a procura do C\&C, porém apenas um pequeno conjunto desses domínios são de fato utilizados. Essa prática gera nomes não legíveis, muitas vezes formados apenas por números e palavras não legíveis. Além disso, por conveniência, é possível que os domínios gerados por uma mesma botnet tenham o mesmo número de caracteres. Ademais, caracteres numéricos são os mais raros e cerca de 90\% dos domínios \textit{.com} e \textit{.net} não contém nenhum número \citep{data2016domain}. Mostrando que o tamanho do domínio de um usuário comum não segue nenhum padrão específico.

Não há garantias de que essa característica seja sempre efetiva, dado que o atacante pode gerar domínios de tamanho variável e evitar números ao gerar o domínio, por isso após a realização dos experimentos será possível confirmar ou não essa hipótese.

Para explorar essas propriedades foram propostas as seguintes características 

\begin{itemize}
\item Quantidade de consultas a domínios com números 
\item Média do comprimento de domínios consultados
\item Desvio Padrão dos comprimentos dos domínios consultados
\end{itemize}

\subsection{Comportamento de Máquina}

O tempo de reação de um humano a uma requisição sem sucesso não pode ser de décimos de segundo, por mera limitação de reflexo. Qualquer sinal de uso que apresente um baixo intervalo entre consultas, algo que apenas uma máquina pode fazer, deve ser considerada suspeita. Além disso, é possível que a máquina acabe por visitar uma quantidade de domínios maior do que o normal ou faça consultas em intervalos regulares, pré-programados, coisa que um ser humano normal raramente irá realizar. Novamente, esse tipo de característica precisa ser validado, pois o comportamento de máquina pode ser mascarado, ao configurar um intervalo aleatório entre consultas e com um certo atraso.

Para explorar essas propriedades foram propostas as seguintes características 

\begin{itemize}
\item Média do intervalo entre as consultas
\item Desvio padrão dos intervalos entre consultas
\item Quantidade total de consultas realizadas
\end{itemize}

\subsection{Domínios Visitados em Comum}
Espera-se que domínios suspeitos sejam acessados apenas por poucas máquinas, como os domínios gerados por algoritmos de geração de domínios para tentar estabelecer uma comunicação com o C\&C. Devido a isso, espera-se que os bots tentem acessar domínios que dificilmente serão procurados por máquinas normais, além do que, se a máquina infectada procura o centro de comando e controle, espera-se que muitos dos domínios consultados por ela também sejam dessa forma, ou seja pouco procurado por outras máquinas.

Para analisar essa propriedade, foi necessário realizar um pré-processamento, que analisa para cada domínio consultado, por quantas máquinas diferentes ele foi consultado. Essa quantidade de máquinas que requisitou um domínio específico chamaremos de grau de requisição do domínio.

Dessa forma, acredita-se que as informações relativas ao grau dos domínios consultados pela máquina podem ser úteis para identificar um comportamento suspeito em uma máquina. Foram levantadas as seguintes características:

\begin{itemize}
\item Grau de requisição mínimo entre os graus dos domínios consultados pela máquina
\item Média dos graus de requisição dos domínios consultados pela máquina
\item Desvio Padrão dos graus de requisição dos domínios consultados pela máquina

\end{itemize}

\subsection{Tipos de Requisição}

Não se tem informação sobre máquinas seguindo padrão quanto ao tipo de requisição DNS solicitada. Porém, esse dado é de fácil acesso e seu estudo pode evidenciar a exploração de alguma fragilidade ainda não analisada nas requisições DNS. Por exemplo, os registros do tipo TXT raramente são utilizados atualmente para leitura de seres humanos. Para isso levantamos algumas características quanto aos tipos de requisição DNS realizadas:

\begin{itemize}
\item Quantidade de consultas do tipo A (Registro de Endereço) realizadas
\item Quantidade de consultas do tipo MX (Registro de Troca de Mensagens) realizadas
\item Quantidade de consultas do tipo CNAME (Nome Canônico) realizadas
\item Quantidade de consultas do tipo TXT (Registro de Texto) realizadas
\end{itemize}

Com essas informações calculadas, é possível submeter as informações contidas no log de DNS para serem tratadas por algoritmos de agrupamento. Espera-se que os bots apresentem comportamento diferente de usuários normais, seguindo padrões que não são seguidos por usuários normais, fazendo com que eles fiquem agrupados em grupos menores, que devem ser investigados.

\chapter{Resultados e Discussões}\label{ch:discussion}

Os experimentos foram realizados com a base de dados organizada como descrita no Capítulo \ref{ch:data_preparation}. Os experimentos foram feitos com o apoio da ferramenta \textit{Scikit-Learn} que possui a implementação de diversos algoritmos de aprendizado de máquina e agrupamento, dentre eles K-Médias, que já foi discutido no capítulo 3. Também foi utilizado o pacote \textit{NumPy}, que é uma ferramenta para cálculos matriciais.

Para a preparação das análises foi adicionado ao banco de dados o registro de atividades do servidor DNS do IME de 12 de março de 2012. Nesse registro de atividades há rastros de 3 clientes suspeitos já confirmados como parte integrante de botnets através de inspeções manuais feitas na época que levaram a criação da lista de suspeitos na figura \ref{fig:suspects}.

\begin{figure}[htbp]
\centering
\includegraphics[scale=0.7]{ip_suspects}
\caption[Lista de IPs Suspeitos Detectados Manualmente na Base de Dados]{Lista de IPs Suspeitos Detectados Manualmente na Base de Dados} \label{fig:suspects}
\end{figure}

Foi realizada a comparação da qualidade da saída do algoritmo para dados normalizados e não normalizados. Para normalizar foi utilizada a forma \ref{eq:norm}

\begin{equation} \label{eq:norm}
\frac{\mathbf{X} - \mathbf{\bar{X}}}{\sigma}
\end{equation}

\section{Análise do Número de Grupos}

Para o cálculo da distribuição da função custo em relação ao número grupos, apresentado na figura \ref{fig:cost_per_k}, foram criados 10 modelos para cada quantidade de centroides e calculou-se a média das funções de custos dentre cada quantidade de centroide. Esse procedimento é realizado para estabilizar os valores da função custo, já que a função custo pode estar sujeita a flutuação, por outro lado isso não pareceu crítico para esse conjunto de dados.

\begin{figure}[htbp]
\centering
\includegraphics[scale=0.7]{cost_per_k}
\caption[Custo pelo Número de Centroides para Dados Normalizados]{Custo pelo Número de Centroides para Dados Normalizados} \label{fig:cost_per_k}
\end{figure}

Percebe-se certa há ambiguidade quanto ao ponto crítico, mas é possível notar uma leve variação de curvatura no ordenada 4.

\section{Análise da Base de Dados}

Utilizando-se 4 centroides como parâmetro, foram realizadas análises da eficácia do algoritmo de agrupamento ao conjunto de exemplos contidos no banco de dados. Em todos os experimentos o sistema consulta a tabela \textit{clients} do banco de dados e retorna na saída padrão a cardinalidade de cada grupo e os IPs das máquina no menor grupo.

A primeira análise foi realizada observando os seguintes campos:

\begin{itemize}
\item \textit{count\_domain\_with\_numbers, }
\item \textit{average\_domain\_length, }
\item \textit{std\_domain\_length, }
\item \textit{count\_request, }
\item \textit{average\_requisition\_degree, }
\item \textit{std\_requisition\_degree e }
\item \textit{minimum\_requisition\_degree }
\end{itemize}

Os dados foram fornecidos ao algoritmo sem nenhum tratamento posterior, já que os dados já foram tratados previamente. Como resultado obteve-se a saída representada na figura \ref{fig:first_out}. Esse resultado não foi considerado um sucesso, pois o menor grupo não continha nenhum dos 3 suspeitos conhecidos, porém já apresenta um grupo de cardinalidade reduzida conforme era esperado.

\begin{figure}[htbp]
\centering
\includegraphics[scale=0.7]{first_out}
\caption[Resultado do Experimento não Normalizado]{Resultado do Experimento não Normalizado} \label{fig:first_out}
\end{figure}

Para a segunda análise, foi efetuada a normalização das características, para que cada uma tivesse o mesmo impacto para as distâncias utilizadas no algoritmo, independente do intervalo para os valores de cada uma. Após essa alteração, observou-se a saída mostrada na Figura \ref{fig:second_out}. Esse foi considerado um resultado bastante satisfatório, já que duas máquinas já confirmadas como pertencentes a botnets foram detectadas no menor grupo, cujos endereços de IP são \textit{200.213.86.2} e \textit{200.213.86.14}.

\begin{figure}[htbp]
\centering
\includegraphics[scale=0.7]{second_out}
\caption[Resultado do Experimento Normalizado]{Resultado do Experimento Normalizado} \label{fig:second_out}
\end{figure}

A hipótese que se levanta para explicar o sucesso é que apesar de as máquinas com comportamento suspeito divergirem das outras máquinas, por exemplo em quantidade de requisição, a distância marginal entre elas é grande o suficiente para confundir o algoritmo ao tentar agrupa-los.

Além disso observou-se que, ao ser descartado, o campo \textit{std\_requisition\_degree} não influenciou na identificação dos itens do menor grupo, isto é os mesmo 26 elementos permanecem consistentemente no mesmo grupo. Isso reforça a necessidade de ser aplicada uma técnica para realizar a seleção das melhores características.
%\chapter{Cronograma}
Embora o objetivo do trabalho seja o desenvolvimento de um projeto e não pesquisa, foi preciso começar por um intenso estudo do problema que queríamos resolver, ou seja das botnets. Isso é uma etapa importante de um projeto de aprendizado de máquina, pois permite uma melhor identificação de quais características serão mais relevantes.

Em seguida, foi feito um estudo dos algoritmos de agrupamento existentes e os objetivos de cada técnica.

Após esses estudos, começa a implementação do sistema detector de botnets. A primeira etapa foi desenvolver um tratamento automatizado dos dados obtidos pela coleta dos logs DNS, já que o objetivo final é de que esse tratamento seja feito diariamente. Depois foram utilizados algoritmos de agrupamento que usaram os dados tratados para identificar padrões de botnets no log DNS coletado. Por fim, os resultados dessas técnicas foram testados e analisados e servirão de motivação para possíveis refinamentos nos algoritmos.

Os passos finais serão constituídos de: finalizar o cálculo de mais características relevantes, utilizar um método para selecionar quais características são mais importantes, melhorar a informação final fornecida pelo sistema ao utilizar algoritmos de associação e informações dos suspeitos.

Durante essas tarefas, desenvolveremos também os relatórios e apresentações para as seguintes avaliações:

\begin{itemize}  
\item Verificação Especial em Maio,
\item Verificação Corrente em Julho,
\item Verificação Final em Setembro.
\end{itemize}

\begin{landscape}
\begin{figure}
\centering
\begin{ganttchart}[
	hgrid,
	vgrid,
	x unit=1.8cm,
	compress calendar,
    bar/.append style={fill=green!40},
	time slot format=isodate-yearmonth
]{2016-02}{2016-09}
\gantttitlecalendar{year, month=shortname} \\
\ganttbar{Estudo de Botnets}{2016-02}{2016-02} \\
\ganttbar{Estudo de Agrupamento}{2016-02}{2016-03} \\
\ganttbar{Escolha das possíveis características}{2016-02}{2016-03} \\
\ganttbar{Preparação dos Dados}{2016-03}{2016-04} \\
\ganttbar{Preparar Relatório e Apresentação VE}{2016-04}{2016-05} \\
\ganttbar{Selecionar e Implementar Algoritmos de Agrupamento}{2016-06}{2016-07} \\
\ganttbar{Preparar Relatório e Apresentação VC}{2016-07}{2016-07} \\
\ganttbar{Teste e Análise de Resultados}{2016-08}{2016-09} \\
\ganttbar{Refinar Algoritmos}{2016-08}{2016-09} \\
\ganttbar{Preparar Relatório e Apresentação VF}{2016-09}{2016-09} \\
\end{ganttchart}
\caption[Cronograma]{Cronograma} \label{fig:cronograma}
\end{figure}
\end{landscape}
\chapter{Conclusão}
Após os estudos apresentados, concluiu-se que pela raridade de ocorrência e tendência anomalia nos comportamentos, o melhor tipo de modelo seria Detecção de Anomalia. Atualmente, há base de dados IPs infectados etiquetados manualmente, porém é possível que se migre para o cenário Semi-Supervisionado, se for provado que há necessidade de mais dados. A aquisição de amostras é facilitada para este trabalho, pois dispõe-se um servidor de DNS no Instituto Militar de Engenharia que pode fornecer dados.

Além disso dos dados, não há garantias de que o melhor modelo de distribuição de probabilidade poderia ser o Gaussiano. Porém é o escolhido por ser o mais simples. Alternativo a ele, pode-se usar o modelo de Misturas de Gaussianas, caso existam regiões não contíguas de máquinas em operação normal.

A escolha dos modelos de distribuição de probabilidade, a validação das características levantas e a constatação da necessidade de mais amostras fazem parte dos trabalhos futuros.


% -----
% PARTE DE REFERÊCIAS BIBLIOGRÁFICAS DE PFC
%
%  As referências do documento de PFC devem estar no arquivo refs.bib
%  Devem seguir o formato bibtex - ver Manual-Referencias.pdf para mais detalhes.
% -----
\bibliographystyle{pfc}
\bibliography{refs}

% -----
% PARTE DE APÊNDICE DE PFC
%
%  Se o documento de PFC não tiver apêndices REMOVER AS LINHAS ABAIXO
%  Adicionar os arquivos .tex de apêndice ao documento com comando \include{•}
% -----
%\inappendix
%%%
%
% ARQUIVO: apendice.tex
%
% VERSÃO: 1.0
% DATA: Maio de 2016
% AUTOR: Coordenação de Trabalhos Especiais SE/8
% 
%  Arquivo tex de exemplo de apêndice do documento de Projeto de Fim de Curso.
%  Este exemplo traz dois apêndices (dois comandos \chapter{•}). Poderiam ser colocados em arquivos .tex
%  separados. Neste caso, o arquivo main.tex deveria ter um \include{•} para cada arquivo .tex
%
% ---
% DETALHES
%  a. todo apêndice deve começar com \chapter{•}
%  b. usar comando \noindent logo após \chapter{•}
%  c. segue os mesmos DETALHES do arquivo .tex de exemplo de capítulo do documento de Projeto de Fim de Curso
% ---
%%
\chapter{Apêndice Exemplo}
\noindent
Curabitur tortor. Pellentesque nibh. Aenean quam. In scelerisque sem at dolor. Maecenas mattis. Sed convallis tristique sem. Proin ut ligula vel nunc egestas porttitor. Morbi lectus risus, iaculis vel, suscipit quis, luctus non, massa. Fusce ac turpis quis ligula lacinia aliquet. Mauris ipsum. Nulla metus metus, ullamcorper vel, tincidunt sed, euismod in, nibh. Quisque volutpat condimentum velit.

Class aptent taciti sociosqu ad litora torquent per conubia nostra, per inceptos himenaeos. Nam nec ante. Sed lacinia, urna non tincidunt mattis, tortor neque adipiscing diam, a cursus ipsum ante quis turpis. Nulla facilisi. Ut fringilla. Suspendisse potenti. Nunc feugiat mi a tellus consequat imperdiet. Vestibulum sapien. Proin quam. Etiam ultrices. Suspendisse in justo eu magna luctus suscipit. Sed lectus. Integer euismod lacus luctus magna.

Lorem ipsum dolor sit amet, consectetur adipiscing elit. Integer nec odio. Praesent libero. Sed cursus ante dapibus diam. Sed nisi. Nulla quis sem at nibh elementum imperdiet. Duis sagittis ipsum. Praesent mauris. Fusce nec tellus sed augue semper porta. Mauris massa. Vestibulum lacinia arcu eget nulla. Class aptent taciti sociosqu ad litora torquent per conubia nostra, per inceptos himenaeos. Curabitur sodales ligula in libero. Sed dignissim lacinia nunc.

\chapter{Apêndice Exemplo 02}
\noindent
Curabitur tortor. Pellentesque nibh. Aenean quam. In scelerisque sem at dolor. Maecenas mattis. Sed convallis tristique sem. Proin ut ligula vel nunc egestas porttitor. Morbi lectus risus, iaculis vel, suscipit quis, luctus non, massa. Fusce ac turpis quis ligula lacinia aliquet. Mauris ipsum. Nulla metus metus, ullamcorper vel, tincidunt sed, euismod in, nibh. Quisque volutpat condimentum velit.

Class aptent taciti sociosqu ad litora torquent per conubia nostra, per inceptos himenaeos. Nam nec ante. Sed lacinia, urna non tincidunt mattis, tortor neque adipiscing diam, a cursus ipsum ante quis turpis. Nulla facilisi. Ut fringilla. Suspendisse potenti. Nunc feugiat mi a tellus consequat imperdiet. Vestibulum sapien. Proin quam. Etiam ultrices. Suspendisse in justo eu magna luctus suscipit. Sed lectus. Integer euismod lacus luctus magna.

Lorem ipsum dolor sit amet, consectetur adipiscing elit. Integer nec odio. Praesent libero. Sed cursus ante dapibus diam. Sed nisi. Nulla quis sem at nibh elementum imperdiet. Duis sagittis ipsum. Praesent mauris. Fusce nec tellus sed augue semper porta. Mauris massa. Vestibulum lacinia arcu eget nulla. Class aptent taciti sociosqu ad litora torquent per conubia nostra, per inceptos himenaeos. Curabitur sodales ligula in libero. Sed dignissim lacinia nunc.

%\outappendix

% -----
% PARTE DE ANEXO DE PFC
%
%  Se o documento de PFC não tiver anexos REMOVER AS LINHAS ABAIXO
%  Adicionar os arquivos .tex de anexo ao documento com comando \include{•}
% -----
%\inannex
%%%
%
% ARQUIVO: anexo.tex
%
% VERSÃO: 1.0
% DATA: Maio de 2016
% AUTOR: Coordenação de Trabalhos Especiais SE/8
% 
%  Arquivo tex de exemplo de anexo do documento de Projeto de Fim de Curso.
%  Este exemplo traz dois anexos (dois comandos \chapter{•}). Poderiam ser colocados em arquivos .tex
%  separados. Neste caso, o arquivo main.tex deveria ter um \include{•} para cada arquivo .tex
%
% ---
% DETALHES
%  a. todo anexo deve começar com \chapter{•}
%  b. usar comando \noindent logo após \chapter{•}
%  c. segue os mesmos DETALHES do arquivo .tex de exemplo de capítulo do documento de Projeto de Fim de Curso
% ---
%%
\chapter{Anexo Exemplo}
\noindent
Id magna feugiat. Erat pellentesque sapien in rhoncus dolor augue vel eget. Erat nibh animi ultricies sit rhoncus. Eleifend aliquam luctus sem turpis habitasse. Lectus arcu ut mi nulla luctus facilisis cursus suspendisse class sociis metus vitae leo consequat lorem ullamcorper arcu. Nunc justo aliquam. Quidem volutpat urna. Nonummy nulla blandit donec vitae ultrices. Netus aliquam vivamus. Vehicula libero leo. Vestibulum consectetuer magna. Sapien aliquam arcu netus etiam lectus. Venenatis tristique morbi non nulla tortor commodo gravida ac neque lacinia urna. Elit mauris adipisci. Vitae sed curabitur. Tellus nunc lectus. Nonummy et integer.

Lorem dictumst enim. Dui vestibulum quisque. Dolor posuere risus. Nullam vitae est magnis est tortor metus dolor integer. Massa elit nec euismod et lacus quam ac malesuada est suspendisse ut est pellentesque vivamus lorem amet non vulputate maecenas et id ultrices lacus odio morbi vitae ac aenean in feugiat elit sodales congue proin dui leo bibendum scelerisque faucibus in suscipit. Nulla parturient in. Eget habitasse fringilla. Eget donec excepturi wisi lorem lacinia. Elementum lorem sem. Pede metus sit. Aenean facilisi pellentesque. Purus dictum ante. Neque amet sed.

Sed leo molestie. Elit fusce placerat lectus quis aliquam nulla turpis platea. Integer mus bibendum sed wisi pretium ullamcorper nunc arcu. Ipsum maecenas sed. Et pariatur in. Ut wisi non. Bibendum nec et quisque quam diam sed dolor lorem. Pellentesque fames donec senectus nulla purus dui nibh praesent. Pariatur nulla augue sapien elit imperdiet aliquam ullamcorper orci. Integer nec mauris. Sit magnis vel ut leo a sapien proin at. Etiam sem aliquam bibendum mauris purus ac sagittis ultrices. Mollis eleifend est. Nec vitae posuere at arcu purus. In elementum vehicula.

\chapter{Anexo Exemplo 02}
\noindent
Id magna feugiat. Erat pellentesque sapien in rhoncus dolor augue vel eget. Erat nibh animi ultricies sit rhoncus. Eleifend aliquam luctus sem turpis habitasse. Lectus arcu ut mi nulla luctus facilisis cursus suspendisse class sociis metus vitae leo consequat lorem ullamcorper arcu. Nunc justo aliquam. Quidem volutpat urna. Nonummy nulla blandit donec vitae ultrices. Netus aliquam vivamus. Vehicula libero leo. Vestibulum consectetuer magna. Sapien aliquam arcu netus etiam lectus. Venenatis tristique morbi non nulla tortor commodo gravida ac neque lacinia urna. Elit mauris adipisci. Vitae sed curabitur. Tellus nunc lectus. Nonummy et integer.

Lorem dictumst enim. Dui vestibulum quisque. Dolor posuere risus. Nullam vitae est magnis est tortor metus dolor integer. Massa elit nec euismod et lacus quam ac malesuada est suspendisse ut est pellentesque vivamus lorem amet non vulputate maecenas et id ultrices lacus odio morbi vitae ac aenean in feugiat elit sodales congue proin dui leo bibendum scelerisque faucibus in suscipit. Nulla parturient in. Eget habitasse fringilla. Eget donec excepturi wisi lorem lacinia. Elementum lorem sem. Pede metus sit. Aenean facilisi pellentesque. Purus dictum ante. Neque amet sed.

Sed leo molestie. Elit fusce placerat lectus quis aliquam nulla turpis platea. Integer mus bibendum sed wisi pretium ullamcorper nunc arcu. Ipsum maecenas sed. Et pariatur in. Ut wisi non. Bibendum nec et quisque quam diam sed dolor lorem. Pellentesque fames donec senectus nulla purus dui nibh praesent. Pariatur nulla augue sapien elit imperdiet aliquam ullamcorper orci. Integer nec mauris. Sit magnis vel ut leo a sapien proin at. Etiam sem aliquam bibendum mauris purus ac sagittis ultrices. Mollis eleifend est. Nec vitae posuere at arcu purus. In elementum vehicula.

%\outannex

% -----
% FIM DO DOCUMENTO DE PFC
% -----
\label{theend}
\end{document}
