\begin{resumo}
	Botnets são uma ameaça cibernética que já trouxe muito prejuízo\cite{silva2013botnets}. Essa ameaça utiliza computadores infectados para realizar atividades fraudulentas como servir páginas piratas para roubar informações sensíveis, enviar \textit{spam} para usuários comuns e enviar sucessivas requisições para derrubar servidores. Por ser uma atividade ilegal, os criminosos realizam a comunicação entre as máquinas com comportamentos divergentes.
	
	Baseado nessa premissa, esse projeto se propõe a detectar máquinas que pertencem a botnets a partir de algoritmos orientados à Detecção de Anomalia. 
	
	Dentro do contexto Exército Brasileiro, o projeto auxiliará a Inteligência do Exército Brasileiro na prevenção de ataques por botnets. Esse sistema será o raciocinador para um sistema que desarticula botnets.

	\vspace{\onelineskip}
	\noindent
	\textbf{Palavras-chave}: botnets, bots, detecção de botnets, aprendizagem de máquinas, clustering, detecção de anomalia.
\end{resumo}

\begin{resumo}[Abstract]
	\begin{otherlanguage*}{english}
		Botnets are a cyber threat that already brought plenty of money dispend\cite{silva2013botnets}. This threat uses infected computers to perform fraudulent activities, such as serving pirated sites to steal sensible information, sending spam to common users and sending successive requests to get servers down. Because it is an illegal activity, the criminals do the communication between the machines with divergent behaviours.
		
		Based on that premise, this project proposes to detect machines that are part of a botnet using Anomaly Detection oriented algorithm.

		Inside the Brazilian Army context, the project will help the Inteligence of Brazilian Army preventing botnet attacks. This system will be the Decision Maker for a botnet desarticulating system.

		\vspace{\onelineskip}
		\noindent
		\textbf{Keywords}: botnets, bots, botnet detection, machine learning, clustering, anomaly detection.
	\end{otherlanguage*}
\end{resumo}
